% !TEX program = xelatex
\documentclass[aspectratio=169, 11pt]{beamer}

% ============================================================
% Packages
% ============================================================
\usepackage{amsmath, amssymb, amsfonts}
\usepackage{graphicx}
\usepackage{tikz}
\usetikzlibrary{shapes, arrows, positioning, calc, decorations.pathreplacing}
\usepackage{booktabs}
\usepackage{bm}
\usepackage{physics}
\usepackage{xcolor}
\usepackage{hyperref}

% ============================================================
% Theme and Colors
% ============================================================
\usetheme{Madrid}
\usecolortheme{seahorse}

\definecolor{darkblue}{RGB}{0, 51, 102}
\definecolor{lightblue}{RGB}{173, 216, 230}
\definecolor{accentred}{RGB}{178, 34, 34}
\definecolor{accentgreen}{RGB}{34, 139, 34}

\setbeamercolor{frametitle}{fg=darkblue}
\setbeamercolor{title}{fg=darkblue}
\setbeamercolor{structure}{fg=darkblue}
\setbeamercolor{block title}{bg=darkblue, fg=white}
\setbeamercolor{block body}{bg=lightblue!30}

% ============================================================
% Title Information
% ============================================================
\title[CTMRG]{\textbf{Corner Transfer Matrix Renormalization Group}}
\subtitle{an Algorithm for 2D Classical Lattice Models\\a Literature Digestion of T. Nishino and K. Okunishi's article (1996).}
\author{Chengzhi Ye}
\institute{ENS de Lyon}
\date{\today}

% ============================================================
% Custom Commands
% ============================================================
\newcommand{\highlight}[1]{\textcolor{accentred}{\textbf{#1}}}
\newcommand{\goodpoint}[1]{\textcolor{accentgreen}{\textbf{#1}}}

% ============================================================
% Document Begin
% ============================================================
\begin{document}

% ------------------------------------------------------------
% Title Slide
% ------------------------------------------------------------
\begin{frame}
    \titlepage
\end{frame}

% ------------------------------------------------------------
% Outline
% ------------------------------------------------------------
\begin{frame}{Outline}
    \tableofcontents
\end{frame}

% ============================================================
% PART 1: From 1D to 2D - The Exponential Wall
% ============================================================
\section{From 1D to 2D: The Contraction Issue}

% ------------------------------------------------------------
% Slide 1.1: 1D Ising Review
% ------------------------------------------------------------
\begin{frame}{Review: 1D Transfer Matrix (Audience's Pre-knowledge)}
    
    \begin{columns}
        \begin{column}{0.5\textwidth}
            \textbf{1D Ising Model:}
            \[
            H = -J \sum_i \sigma_i \sigma_{i+1}
            \]
            
            \vspace{0.3cm}
            
            \textbf{Partition function as contraction:}
            \[
            Z = \text{Tr}(T \cdot T \cdot T \cdots T) = \text{Tr}(T^N)
            \]
            
            \vspace{0.2cm}
            
            where $T \in \mathbb{R}^{2 \times 2}$ encodes local interactions.
            
            \vspace{0.3cm}
            
            \textbf{Local observable:}
            \[
            \langle \sigma_k \rangle = \frac{\text{Tr}(T^{k} \cdot f(\uparrow,\downarrow) \cdot T^{N-k})}{\text{Tr}(T^N)}
            \]
        \end{column}
        
        \begin{column}{0.5\textwidth}
            % === FIGURE: 1D chain with transfer matrices ===
            \begin{center}
                \includegraphics[width=0.95\textwidth]{figures/fig_1d_transfer_chain.pdf}
            \end{center}
        \end{column}
    \end{columns}
    
    \vspace{0.2cm}
    \begin{block}{Key Point}
        \textbf{Translational invariance:} All $T$ are identical $\Rightarrow$ they \goodpoint{commute}!
        
        Huge contraction bypassed: $T^N = U \Lambda^N U^\dagger$ with single diagonalization.
    \end{block}
\end{frame}

% ------------------------------------------------------------
% Slide 1.2: When 1D Becomes Hard
% ------------------------------------------------------------
\begin{frame}{When Even 1D Classical Needs Renormalization Group}
    
    \textbf{Recall:} For uniform 1D Ising, $Z = \text{Tr}(T^N)$ has no contraction issues.
    
    \vspace{0.3cm}
    
    \begin{columns}
        \begin{column}{0.5\textwidth}
            \textbf{But what if...}
            \begin{itemize}
                \item Couplings $J_i$ are \highlight{site-dependent}?
                \item Random disorder: $J_i \sim \mathcal{N}(\bar{J}, \sigma)$?
                \item Open boundary conditions (no translation symmetry)?
            \end{itemize}
            
            \vspace{0.3cm}
            
            \textbf{Then:} Cannot diagonalize $T$ once!
            \[
            Z = \text{Tr}(T_1 \cdot T_2 \cdot T_3 \cdots T_N)
            \]
            Each $T_i$ is different $\Rightarrow$ no simple $\lambda^N$ formula.
        \end{column}
        
        \begin{column}{0.5\textwidth}
            % === FIGURE: 1D chain with varying couplings ===
            \begin{center}
                \includegraphics[width=0.95\textwidth]{figures/fig_1d_varying_couplings.pdf}
            \end{center}
            
            \begin{alertblock}{The Problem}
                For $N$ sites: need $O(N)$ matrix multiplications.
                
                Not exponential, but \highlight{no closed-form solution}.
            \end{alertblock}
        \end{column}
    \end{columns}
\end{frame}

% ------------------------------------------------------------
% Slide 1.3: 1D Solution - Transfer Matrix RG
% ------------------------------------------------------------
\begin{frame}{Solution: Transfer Matrix Renormalization Group (TMRG)}
    
    \begin{center}
        \textbf{Insight:} Not all configurations contribute equally to $Z$ $\Rightarrow$ Keep only the \goodpoint{most relevant} ones!
    \end{center}
    
    
    \begin{columns}
    	\vspace{1.8cm}
        \begin{column}{0.6\textwidth}
            % === FIGURE: TMRG environments ===
            \begin{center}
                \includegraphics[width=0.8\textwidth]{figures/fig_1d_tmrg_environments.png}
            \end{center}
            
            
            \vspace{0.1cm}
            
            {\small \textit{$\chi$ = bond dimension (most relevant configurations in compressed basis)}}
        \end{column}
        
        \begin{column}{0.5\textwidth}
            \textbf{TMRG Algorithm:}
            \begin{enumerate}
                \item \textbf{Grow:} $L' = L \cdot T~,~R'=T\cdot R$ 
                
                {\small (add sites)}
                
                \item \textbf{Truncate:} SVD $\to$ keep $\chi$ largest
                
                {\small (coarse-grain)}
            \end{enumerate}
            \vspace{0.15cm}
            
            \begin{block}{RG Fixed Point}
            Iterate until $L^*, R^*$ converge $\Rightarrow$ Thermodynamic limit!
            \end{block}
            
            \vspace{0.3cm}
            
            \textit{Kind-of like infinite-DMRG by adding new pair of sites}
        \end{column}
    \end{columns}
\end{frame}

% ============================================================
% PART 2: Going to 2D - The Network Problem
% ============================================================
\section{Going to 2D: A Fundamentally Harder Problem}

% ------------------------------------------------------------
% Slide 2.1: 2D is a Network!
% ------------------------------------------------------------
\begin{frame}{2D: The Transfer Matrix Becomes a \highlight{Network}}
    
    \begin{columns}
        \begin{column}{0.45\textwidth}
            \textbf{1D:} Ordered product of matrices
            \[
            Z = \text{Tr}(T_1 \cdot T_2 \cdots T_N)
            \]
            Contract left-to-right: $O(N)$.
            
            \vspace{0.5cm}
            
            \textbf{2D:} $T$ still $2 \times 2$ on each \highlight{bond},
            
            but bonds form a \highlight{2D network}!
            
            $\Rightarrow$ No natural ordering.
        \end{column}
        
        \begin{column}{0.55\textwidth}
            % === FIGURE: 2D tensor network (45° tilted) ===
            \begin{center}
                \includegraphics[width=\textwidth]{figures/fig_2d_tensor_network.pdf}
            \end{center}
        \end{column}
    \end{columns}
    
    \vspace{-0.8cm}
    
    \begin{alertblock}{The Fundamental Problem}
        \highlight{No natural contraction order!} The 2D tensor network cannot be reduced to a simple trace of matrix products. \textbf{Exact contraction is way too slow!}
    \end{alertblock}
\end{frame}

% ------------------------------------------------------------
% Slide 2.2: One Approach - Row-to-Row (Brief)
% ------------------------------------------------------------
\begin{frame}{One Approach: Row-to-Row Transfer (DMRG-style)}
    
    \textbf{Idea:} Group one row of $L$ spins $\to$ treat as a ``super-spin'' with $2^L$ states.
    
    \vspace{0.3cm}
    
    \begin{columns}
        \begin{column}{0.5\textwidth}
            % === FIGURE PLACEHOLDER ===
            % Concept: Row grouped as super-spin
            %
            % ASCII draft:
            %   Row n:   ○──○──○──○──○  →  [σ_row] (2^L states)
            %            │  │  │  │  │
            %   Row n+1: ○──○──○──○──○  →  [σ'_row]
            %
            %   T_row ∈ R^{2^L × 2^L}
            \begin{center}
                \fbox{\parbox{0.95\textwidth}{
                    \centering
                    \vspace{1.8cm}
                    \textit{[Figure: Row $\to$ super-spin, $T_{\text{row}} \in \mathbb{R}^{2^L \times 2^L}$]}
                    \vspace{1.8cm}
                }}
            \end{center}
            
            Then $Z = \text{Tr}(T_{\text{row}}^M)$ looks like 1D!
        \end{column}
        
        \begin{column}{0.5\textwidth}
            \textbf{Apply DMRG/TMRG ideas:}
            \begin{itemize}
                \item Row config space $\to$ MPS
                \item Row-to-row transfer $\to$ MPO
                \item Truncate via SVD
            \end{itemize}
            
            \vspace{0.3cm}
            
            \begin{block}{Pros \& Cons}
                \textcolor{accentgreen}{\textbf{+}} Systematic, well-understood
                
                \textcolor{accentred}{\textbf{--}} Breaks 2D symmetry
                
                \textcolor{accentred}{\textbf{--}} Hard to generalize to other lattices
            \end{block}
            
            \vspace{0.2cm}
            
            \textit{(Details in Appendix)}
        \end{column}
    \end{columns}
\end{frame}

% ------------------------------------------------------------
% Slide 2.3: A More Natural 2D Perspective
% ------------------------------------------------------------
\begin{frame}{A More Natural Approach: Use Environment just like in 1D TMRG}
    
    \begin{center}
        \textbf{Question:} If we only want \highlight{local observables} $\langle \sigma_{i,j} \rangle$, 
        
        do we really need to contract the \textit{entire} infinite lattice?
    \end{center}
    
    \vspace{0.2cm}
    
    \begin{columns}
        \begin{column}{0.5\textwidth}
            % === FIGURE: Local site surrounded by environment ===
            \begin{center}
                \includegraphics[width=\textwidth]{figures/fig_2d_local_environment.pdf}
            \end{center}
        \end{column}
        
        \begin{column}{0.5\textwidth}
            \textbf{Key Insight:}
            
            Decompose the infinite 2D environment into \highlight{geometrically natural} pieces:
            
            \vspace{0.3cm}
            
            \begin{itemize}
                \item \textbf{4 Corners} (quarter-planes)
                \item \textbf{4 Edges} (half-infinite strips)
            \end{itemize}
            
            \vspace{0.3cm}
            
            \begin{block}{Baxter's Insight (1968)}
                The \textbf{Corner Transfer Matrix} encodes a quarter of the infinite plane!
            \end{block}
        \end{column}
    \end{columns}
    
    \vspace{0.3cm}
    
    \begin{center}
        $\Rightarrow$ \textbf{CTMRG:} Apply RG to corners and edges!
    \end{center}
\end{frame}

% ============================================================
% PART 3: CTMRG for 2D
% ============================================================
\section{CTMRG: Extending to 2D}

% ------------------------------------------------------------
% Slide 3.1: 2D Environment Structure
% ------------------------------------------------------------
\begin{frame}{2D Analog: Four Corners + Four Edges}
    
    
    \vspace{-0.2cm}
    
    \begin{columns}
        \begin{column}{0.6\textwidth}
        	\begin{center}
        		\includegraphics[width=0.9\textwidth]{figures/fig_2d_ctmrg_structure.png}
        	\end{center}
        \end{column}
        \begin{column}{0.4\textwidth}
        	\textbf{Corners $C$:}
        	\begin{itemize}
        		\item Quadrant ($\chi \times \chi$)
        	\end{itemize}
            \textbf{Edges $T$:}
            \begin{itemize}
                \item Half-strip ($\chi \times d \times \chi$)
            \end{itemize}
            \textbf{Local tensor $a$:}
            \begin{itemize}
                \item Rank-4 ($d^4$)
            \end{itemize}
        \end{column}
    \end{columns}
    
    \vspace{0.1cm}
    
    \begin{block}{Key: What is $a$?}
        $a$ is a \highlight{rank-4 tensor} --- for computing $Z$: $a = W$ \textit{plaquette tensor}; for $\langle \sigma \rangle$: $a = s_1\delta_{s_1s_2s_3s_4}$, etc.
    \end{block}
\end{frame}

% ------------------------------------------------------------
% Slide 3.2: Local Tensor
% ------------------------------------------------------------
\begin{frame}{The Plaquette Tensor: From Bonds to Faces}
    
    \begin{center}
        \includegraphics[width=0.9\textwidth]{figures/fig_plaquette_tensor.pdf}
    \end{center}
    
    \vspace{-0.3cm}
    
    \begin{columns}
        \begin{column}{0.5\textwidth}
            \textbf{For 2D Ising:}
            \[
            W_{u,d,l,r} = \sum_{\sigma} T_{\sigma,u} T_{\sigma,d} T_{\sigma,l} T_{\sigma,r}
            \]
        \end{column}
        \begin{column}{0.5\textwidth}
            \textbf{Result:} $W$ is the \highlight{rank-4 local tensor}
            
            Each $W$ encodes one \textit{face} of the tilted lattice.
        \end{column}
    \end{columns}
\end{frame}

% ------------------------------------------------------------
% Slide 3.3: CTMRG Grow Step
% ------------------------------------------------------------
\begin{frame}{CTMRG Iteration: Step 1 --- Grow (Absorb)}
    
    \begin{center}
        \includegraphics[width=0.85\textwidth]{figures/fig_ctmrg_grow.pdf}
    \end{center}
\end{frame}

% ------------------------------------------------------------
% Slide 3.4: CTMRG Truncate Step
% ------------------------------------------------------------
\begin{frame}{CTMRG Iteration: Step 2 --- Truncate}
    
    \begin{center}
        \includegraphics[width=0.85\textwidth]{figures/fig_ctmrg_truncate.pdf}
    \end{center}
\end{frame}

% ------------------------------------------------------------
% Slide 3.5: CTMRG Full Algorithm
% ------------------------------------------------------------
\begin{frame}{CTMRG: Complete Algorithm}
    
    \begin{center}
        \includegraphics[width=0.65\textwidth]{figures/fig_ctmrg_algorithm.pdf}
    \end{center}
\end{frame}




% ============================================================
% PART 5: Summary
% ============================================================
\section{Summary}

% ------------------------------------------------------------
% Slide 5.1: Core Steps
% ------------------------------------------------------------
\begin{frame}{Summary: The Core of CTMRG}
    
    \begin{block}{CTMRG in Three Steps}
        \begin{enumerate}
            \item \textbf{Decompose:} Infinite square lattice $\to$ 4 corners + 4 edges (customed for other lattice);
            \item \textbf{Grow and Truncate (RG):} add row+column then absorb, SVD to keep $\chi$ most relevant; 
            \item    $ Z = W_{s_1s_2s_3s_4} \text{Tr}(C_1 T^{s_1}_1 C_2 T^{s_2}_2 C_3 T^{s_3}_3 C_4 T^{s_4}_4)~$;$~~
            \langle \mathcal{O} \rangle = \frac{ \mathcal{O}_{s_1s_2s_3s_4}}{Z}\text{Tr}(C_1 T^{s_1}_1 C_2 T^{s_2}_2 C_3 T^{s_3}_3 C_4 T^{s_4}_4)$
        \end{enumerate}
    \end{block}
    
    \vspace{0.5cm}
    
    % === FIGURE PLACEHOLDER ===
    % Concept: Visual summary of the three steps
    %
    % ASCII draft:
    %   [Decompose]      [Grow]          [Truncate]
    %   4 corners    →   absorb a    →   SVD + keep χ
    %   + 4 edges        (larger)        (back to χ)
    %
    % Show as three-panel diagram
    \begin{center}
        \fbox{\parbox{0.85\textwidth}{
            \centering
            \vspace{2.5cm}
            \textit{[Figure: Visual summary --- Decompose $\to$ Grow $\to$ Truncate]}
            \vspace{2.5cm}
        }}
    \end{center}
\end{frame}

% ============================================================
% Thank You Slide
% ============================================================
\begin{frame}
    \begin{center}
        \Huge \textbf{Thank You All for Your Attention!}
        
        \vspace{1cm}
        
        \Large And thanks to Prof. Tommaso Roscilde, Dr. Fabio Mezzacapo, Filippo Caleca and Saverio Bocini for your teaching and assistance!
    \end{center}
\end{frame}

% ============================================================
% APPENDIX
% ============================================================
\appendix
\section*{Appendix}

% ============================================================
% Appendix B: 1D Transfer Matrix RG (TMRG) - Detailed
% ============================================================

% ------------------------------------------------------------
% Appendix B.1: TMRG Setup
% ------------------------------------------------------------
\begin{frame}{Appendix: TMRG Setup in Detail}
    
    \textbf{Goal:} Compute $Z = \text{Tr}(T_1 \cdot T_2 \cdots T_N)$ for large $N$.
    
    \vspace{0.3cm}
    
    % === FIGURE PLACEHOLDER ===
    \begin{center}
        \fbox{\parbox{0.8\textwidth}{
            \centering
            \vspace{2cm}
            \textit{[Figure: Left environment $L$, local site $T$, right environment $R$]}
            \vspace{2cm}
        }}
    \end{center}
    
    \vspace{0.3cm}
    
    \begin{columns}
        \begin{column}{0.5\textwidth}
            \textbf{Left Environment $L$:}
            \[
            L_{\sigma} = \sum_{\sigma_1, \ldots, \sigma_{k-1}} T_1 \cdot T_2 \cdots T_{k-1}
            \]
            Encodes all configs to the left of site $k$.
        \end{column}
        \begin{column}{0.5\textwidth}
            \textbf{Right Environment $R$:}
            \[
            R_{\sigma} = \sum_{\sigma_{k+1}, \ldots, \sigma_N} T_{k+1} \cdots T_N
            \]
            Encodes all configs to the right of site $k$.
        \end{column}
    \end{columns}
\end{frame}

% ------------------------------------------------------------
% Appendix B.2: TMRG Grow Step
% ------------------------------------------------------------
\begin{frame}{Appendix: TMRG Step 1 --- Grow (Absorption)}
    
    \textbf{Absorption:} Add one more site to the environment.
    
    \vspace{0.3cm}
    
    \begin{center}
        \fbox{\parbox{0.8\textwidth}{
            \centering
            \vspace{2cm}
            \textit{[Figure: $L' = L \cdot T_k$, growing the left environment]}
            \vspace{2cm}
        }}
    \end{center}
    
    \vspace{0.3cm}
    
    \textbf{Mathematically:}
    \[
    L'_{\sigma_k} = \sum_{\sigma_{k-1}} L_{\sigma_{k-1}} \cdot (T_k)_{\sigma_{k-1}, \sigma_k}
    \]
    
    \begin{alertblock}{Problem}
        If $L$ has dimension $\chi$, after absorption $L'$ has dimension $\chi \cdot d$.
        
        After $N$ absorptions: dimension $\sim d^N$ --- \highlight{exponential}!
    \end{alertblock}
\end{frame}

% ------------------------------------------------------------
% Appendix B.3: TMRG Truncate Step
% ------------------------------------------------------------
\begin{frame}{Appendix: TMRG Step 2 --- Truncate (SVD)}
    
    \textbf{Idea:} Compress $L'$ back to dimension $\chi$ using SVD.
    
    \vspace{0.3cm}
    
    \begin{columns}
        \begin{column}{0.5\textwidth}
            \textbf{Form the ``density matrix'':}
            \[
            \rho = L' \cdot R'^T
            \]
            
            \textbf{SVD:}
            \[
            \rho = U \Sigma V^\dagger
            \]
            
            \textbf{Truncate:}
            \[
            P = U_{:, 1:\chi}
            \]
            
            \textbf{New environment:}
            \[
            L_{\text{new}} = P^\dagger L'
            \]
        \end{column}
        
        \begin{column}{0.5\textwidth}
            \begin{center}
                \fbox{\parbox{0.95\textwidth}{
                    \centering
                    \vspace{2cm}
                    \textit{[Figure: SVD spectrum, keep $\chi$ largest]}
                    \vspace{2cm}
                }}
            \end{center}
            
            \begin{block}{Eckart-Young Theorem}
                SVD gives the \textbf{optimal} rank-$\chi$ approximation in Frobenius norm.
            \end{block}
        \end{column}
    \end{columns}
\end{frame}

% ------------------------------------------------------------
% Appendix B.4: TMRG Convergence
% ------------------------------------------------------------
\begin{frame}{Appendix: TMRG Convergence and Observables}
    
    \textbf{Iterate:} Grow $\to$ Truncate $\to$ Grow $\to$ Truncate $\to \cdots$
    
    \vspace{0.3cm}
    
    \begin{columns}
        \begin{column}{0.5\textwidth}
            \textbf{Convergence criterion:}
            
            Fixed point: $L^*, R^*$ such that
            \[
            L^* \xrightarrow{\text{grow+truncate}} L^*
            \]
            
            \vspace{0.3cm}
            
            \textbf{Free energy:}
            \[
            f = -k_B T \lim_{N\to\infty} \frac{1}{N} \ln Z
            \]
            \[
            = -k_B T \ln \sigma_1^*
            \]
        \end{column}
        
        \begin{column}{0.5\textwidth}
            \textbf{Local observables:}
            \[
            \langle \sigma_k \rangle = \frac{L^* \cdot \sigma_k \cdot R^*}{L^* \cdot R^*}
            \]
            
            \textbf{Correlation functions:}
            \[
            \langle \sigma_i \sigma_j \rangle = \frac{L^* \cdot \sigma_i \cdot T^{|i-j|} \cdot \sigma_j \cdot R^*}{Z}
            \]
            
            \vspace{0.3cm}
            
            \begin{block}{Physical Meaning}
                $L^*, R^*$ encode the \textbf{thermodynamic limit}.
            \end{block}
        \end{column}
    \end{columns}
\end{frame}

% ============================================================
% Appendix C: 2D Row-to-Row DMRG - Detailed
% ============================================================

% ------------------------------------------------------------
% Appendix C.1: Row as Super-Spin
% ------------------------------------------------------------
\begin{frame}{Appendix: 2D Row-to-Row --- Row as Super-Spin}
    
    \textbf{Idea:} Treat one row of $L$ spins as a single ``super-spin''.
    
    \vspace{0.3cm}
    
    \begin{columns}
        \begin{column}{0.5\textwidth}
            \begin{center}
                \fbox{\parbox{0.95\textwidth}{
                    \centering
                    \vspace{2cm}
                    \textit{[Figure: Row $\to$ super-spin with $2^L$ states]}
                    \vspace{2cm}
                }}
            \end{center}
        \end{column}
        
        \begin{column}{0.5\textwidth}
            \textbf{Row configuration:}
            \[
            \vec{\sigma} = (\sigma_1, \sigma_2, \ldots, \sigma_L)
            \]
            
            Total states: $2^L$.
            
            \vspace{0.3cm}
            
            \textbf{Row-to-row transfer:}
            \[
            (T_{\text{row}})_{\vec{\sigma}, \vec{\sigma}'} = \prod_{i=1}^L e^{\beta J \sigma_i \sigma'_i} \prod_{i=1}^{L-1} e^{\beta J \sigma_i \sigma_{i+1}}
            \]
            
            $T_{\text{row}} \in \mathbb{R}^{2^L \times 2^L}$.
        \end{column}
    \end{columns}
    
    \vspace{0.3cm}
    
    \begin{block}{Key Point}
        Now $Z = \text{Tr}(T_{\text{row}}^M)$ looks like 1D problem with $2^L$-dimensional transfer matrix!
    \end{block}
\end{frame}

% ------------------------------------------------------------
% Appendix C.2: MPO Decomposition
% ------------------------------------------------------------
\begin{frame}{Appendix: Row Transfer Matrix = MPO}
    
    \textbf{Key insight:} $T_{\text{row}}$ has a \highlight{tensor network} structure!
    
    \vspace{0.3cm}
    
    \begin{columns}
        \begin{column}{0.55\textwidth}
            \textbf{Decompose} $T_{\text{row}}$ into local tensors:
            \[
            T_{\vec{\sigma}, \vec{\sigma}'} = \sum_{\alpha_1, \ldots} W^{\sigma_1 \sigma'_1}_{\alpha_0 \alpha_1} W^{\sigma_2 \sigma'_2}_{\alpha_1 \alpha_2} \cdots
            \]
            
            \begin{center}
                \fbox{\parbox{0.95\textwidth}{
                    \centering
                    \vspace{1.5cm}
                    \textit{[Figure: MPO structure of $T_{\text{row}}$]}
                    \vspace{1.5cm}
                }}
            \end{center}
        \end{column}
        
        \begin{column}{0.45\textwidth}
            \textbf{Local tensor $W$:}
            \[
            W^{\sigma \sigma'}_{\alpha \beta} = \text{local Boltzmann weight}
            \]
            
            \begin{itemize}
                \item $\sigma, \sigma'$: spins in rows $n$, $n+1$
                \item $\alpha, \beta$: auxiliary (horizontal bonds)
            \end{itemize}
            
            \vspace{0.2cm}
            
            \begin{block}{No Quantum!}
                ``MPO'' is just a factorization of the classical transfer matrix.
            \end{block}
        \end{column}
    \end{columns}
\end{frame}

% ------------------------------------------------------------
% Appendix C.3: MPS for Row Configuration
% ------------------------------------------------------------
\begin{frame}{Appendix: Boundary = MPS}
    
    \textbf{MPS:} Efficient representation of row configuration space.
    
    \vspace{0.3cm}
    
    \begin{columns}
        \begin{column}{0.5\textwidth}
            \textbf{Full vector:}
            \[
            |R\rangle = \sum_{\vec{\sigma}} R_{\vec{\sigma}} |\vec{\sigma}\rangle
            \]
            has $2^L$ components.
            
            \vspace{0.3cm}
            
            \textbf{MPS compression:}
            \[
            R_{\vec{\sigma}} = A^{\sigma_1} A^{\sigma_2} \cdots A^{\sigma_L}
            \]
            
            Only $L \cdot \chi^2 \cdot 2$ parameters!
        \end{column}
        
        \begin{column}{0.5\textwidth}
            \begin{center}
                \fbox{\parbox{0.95\textwidth}{
                    \centering
                    \vspace{1.8cm}
                    \textit{[Figure: MPS tensor network]}
                    \vspace{1.8cm}
                }}
            \end{center}
            
            \begin{block}{Physical Meaning}
                MPS compresses $2^L$ row configs into $\chi$ effective states with \textbf{limited entanglement}.
            \end{block}
        \end{column}
    \end{columns}
\end{frame}

% ------------------------------------------------------------
% Appendix C.4: Row-to-Row DMRG Algorithm
% ------------------------------------------------------------
\begin{frame}{Appendix: Row-to-Row DMRG Algorithm}
    
    \textbf{Combine:} MPS (boundary) + MPO (transfer) + SVD (truncation).
    
    \vspace{0.3cm}
    
    \begin{enumerate}
        \item Initialize MPS $|L\rangle$, $|R\rangle$ for boundaries
        
        \item \textbf{Grow:} Apply MPO $T_{\text{row}}$ to boundaries
        \[
        |L'\rangle = T_{\text{row}} |L\rangle
        \]
        
        \item \textbf{Truncate:} SVD to compress bond dimension back to $\chi$
        
        \item Iterate until convergence to fixed point $|L^*\rangle$, $|R^*\rangle$
    \end{enumerate}
    
    \vspace{0.3cm}
    
    \begin{columns}
        \begin{column}{0.5\textwidth}
            \begin{block}{Pros}
                \begin{itemize}
                    \item Systematic, well-understood
                    \item Controlled approximation
                \end{itemize}
            \end{block}
        \end{column}
        \begin{column}{0.5\textwidth}
            \begin{alertblock}{Cons}
                \begin{itemize}
                    \item Breaks 2D rotational symmetry
                    \item Hard to generalize to other lattices
                \end{itemize}
            \end{alertblock}
        \end{column}
    \end{columns}
\end{frame}

% ============================================================
% Appendix A: Additional Topics
% ============================================================

% ------------------------------------------------------------
% Appendix A.1: No Sign Problem
% ------------------------------------------------------------
\begin{frame}{Appendix: No Monte Carlo Sign Problem}
    
    \textbf{Classical models:} All Boltzmann weights are \highlight{positive}!
    
    \[
    W = e^{-\beta H} > 0 \quad \text{always}
    \]
    
    \vspace{0.3cm}
    
    % === FIGURE PLACEHOLDER ===
    % Concept: Comparison of classical vs quantum MC
    %
    % ASCII draft:
    %   Classical:  W > 0  →  No sign problem  →  MC works!
    %   Quantum:    W can be < 0 (fermions, frustration)
    %                      →  Sign problem  →  MC fails
    \begin{center}
        \fbox{\parbox{0.7\textwidth}{
            \centering
            \vspace{2cm}
            \textit{[Figure: Classical (positive) vs Quantum (sign problem)]}
            \vspace{2cm}
        }}
    \end{center}
    
    \vspace{0.3cm}
    
    \begin{block}{Advantage of Tensor Network Methods}
        \begin{itemize}
            \item No statistical sampling $\Rightarrow$ no sign problem issue
            \item Deterministic algorithm with controlled error (via $\chi$)
            \item Works equally well for frustrated/quantum systems
        \end{itemize}
    \end{block}
\end{frame}

% ------------------------------------------------------------
% Appendix A.2: Why Not 3D?
% ------------------------------------------------------------
\begin{frame}{Appendix: Why is 3D Difficult?}
    
    \textbf{In 2D:} Environment tensors are \highlight{1D objects} (edges).
    
    \textbf{In 3D:} Environment would be \highlight{2D surfaces} --- back to exponential!
    
    \vspace{0.3cm}
    
    % === FIGURE PLACEHOLDER ===
    % Concept: 2D boundary in 3D vs 1D boundary in 2D
    %
    % ASCII draft:
    %   2D system:  boundary is 1D line (manageable with χ)
    %   3D system:  boundary is 2D surface (exponential again!)
    %
    % Show schematic of 3D cube with 2D boundary
    \begin{center}
        \fbox{\parbox{0.75\textwidth}{
            \centering
            \vspace{2.5cm}
            \textit{[Figure: 2D boundary problem in 3D systems]}
            \vspace{2.5cm}
        }}
    \end{center}
    
    \vspace{0.3cm}
    
    \begin{alertblock}{The 3D Challenge}
        No simple analog of CTMRG for 3D --- active research area!
        
        Possible approaches: Higher-order tensor RG, Monte Carlo + TN hybrids
    \end{alertblock}
\end{frame}

% ------------------------------------------------------------
% Appendix A.3: Extensions
% ------------------------------------------------------------
\begin{frame}{Appendix: Extensions and Generalizations}
    
    \textbf{Beyond square lattice Ising:}
    
    \vspace{0.3cm}
    
    \begin{columns}
        \begin{column}{0.5\textwidth}
            \textbf{Different lattices:}
            \begin{itemize}
                \item Honeycomb
                \item Triangular
                \item Kagome
            \end{itemize}
            
            \vspace{0.3cm}
            
            \textbf{Different models:}
            \begin{itemize}
                \item Potts model
                \item Clock model
                \item Vertex models
            \end{itemize}
        \end{column}
        
        \begin{column}{0.5\textwidth}
            \textbf{Quantum systems (via iPEPS):}
            \begin{itemize}
                \item 2D Heisenberg model
                \item Frustrated magnets
                \item Topological phases
            \end{itemize}
            
            \vspace{0.3cm}
            
            \textbf{Improvements:}
            \begin{itemize}
                \item Directional CTMRG
                \item Full-update vs simple-update
                \item Gradient optimization
            \end{itemize}
        \end{column}
    \end{columns}
\end{frame}


% ------------------------------------------------------------
% Slide 5.2: DMRG vs CTMRG
% ------------------------------------------------------------

\begin{frame}{Comparison: iDMRG vs CTMRG}
	
	\begin{table}
		\centering
		\renewcommand{\arraystretch}{1.3}
		\begin{tabular}{l|c|c}
			\toprule
			\textbf{Aspect} & \textbf{iDMRG (1D)} & \textbf{CTMRG (2D)} \\
			\midrule
			Dimension & 1D chain & 2D square lattice \\
			\midrule
			Environment & Left + Right & 4 Corners + 4 Edges \\
			\midrule
			Grow step & Add site pair & Add row + column \\
			\midrule
			Truncation & SVD on center bond & SVD on corner boundaries \\
			\midrule
			Bond dimension & $\chi$ (MPS) & $\chi$ (environment) \\
			\midrule
			Fixed point & $L^*, R^*$ & $C_i^*, T_i^*$ \\
			\midrule
			Computational cost & $O(\chi^3)$ & $O(\chi^6)$ or $O(\chi^5)$ \\
			\bottomrule
		\end{tabular}
	\end{table}
	
	\vspace{0.3cm}
	
	\begin{block}{Take-Home Message}
		\goodpoint{Same RG philosophy}, different geometric structure!
	\end{block}
\end{frame}
% ------------------------------------------------------------
% Appendix A.4: Computational Complexity
% ------------------------------------------------------------
\begin{frame}{Appendix: Computational Complexity}
    
    \textbf{CTMRG scaling:}
    
    \begin{table}
        \centering
        \begin{tabular}{l|c|c}
            \toprule
            \textbf{Operation} & \textbf{Cost} & \textbf{Bottleneck?} \\
            \midrule
            Corner absorption & $O(\chi^4 d^2)$ & \\
            Edge absorption & $O(\chi^3 d^2)$ & \\
            Build density matrix & $O(\chi^4)$ & \\
            SVD for projector & $O(\chi^3 d^3)$ & \checkmark \\
            Apply truncation & $O(\chi^3 d)$ & \\
            \midrule
            \textbf{Total per iteration} & $O(\chi^3 d^3)$ to $O(\chi^6)$ & \\
            \bottomrule
        \end{tabular}
    \end{table}
    
    \vspace{0.3cm}
    
    \begin{block}{Comparison}
        \begin{itemize}
            \item iDMRG (1D): $O(\chi^3)$ per sweep
            \item CTMRG (2D): $O(\chi^5)$ -- $O(\chi^6)$ per iteration
            \item More expensive, but still \goodpoint{polynomial} in $\chi$!
        \end{itemize}
    \end{block}
\end{frame}

% ------------------------------------------------------------
% Appendix A.5: Key References
% ------------------------------------------------------------
\begin{frame}{Appendix: Key References}
    
    \textbf{Original works:}
    \begin{itemize}
        \item R. J. Baxter, \textit{J. Math. Phys.} \textbf{9}, 650 (1968) --- Corner transfer matrices
        \item R. J. Baxter, \textit{J. Stat. Phys.} \textbf{19}, 461 (1978) --- CTM method
    \end{itemize}
    
    \vspace{0.3cm}
    
    \textbf{Modern CTMRG:}
    \begin{itemize}
        \item T. Nishino \& K. Okunishi, \textit{J. Phys. Soc. Jpn.} \textbf{65}, 891 (1996)
        \item T. Nishino \& K. Okunishi, \textit{J. Phys. Soc. Jpn.} \textbf{66}, 3040 (1997)
    \end{itemize}
    
    \vspace{0.3cm}
    
    \textbf{CTMRG for iPEPS:}
    \begin{itemize}
        \item R. Orús \& G. Vidal, \textit{Phys. Rev. B} \textbf{80}, 094403 (2009)
        \item P. Corboz et al., \textit{Phys. Rev. B} \textbf{84}, 041108(R) (2011)
    \end{itemize}
    
    \vspace{0.3cm}
    
    \textbf{Reviews:}
    \begin{itemize}
        \item R. Orús, \textit{Ann. Phys.} \textbf{349}, 117 (2014) --- Tensor networks review
    \end{itemize}
\end{frame}

\end{document}
