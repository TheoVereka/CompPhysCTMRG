% !TEX program = xelatex
\documentclass[aspectratio=169, 11pt]{beamer}

% ============================================================
% Packages
% ============================================================
\usepackage{amsmath, amssymb, amsfonts}
\usepackage{graphicx}
\usepackage{tikz}
\usetikzlibrary{shapes, arrows, positioning, calc, decorations.pathreplacing}
\usepackage{booktabs}
\usepackage{bm}
\usepackage{physics}
\usepackage{xcolor}
\usepackage{hyperref}

% ============================================================
% Theme and Colors
% ============================================================
\usetheme{Madrid}
\usecolortheme{seahorse}

\definecolor{darkblue}{RGB}{0, 51, 102}
\definecolor{lightblue}{RGB}{173, 216, 230}
\definecolor{accentred}{RGB}{178, 34, 34}
\definecolor{accentgreen}{RGB}{34, 139, 34}

\setbeamercolor{frametitle}{fg=darkblue}
\setbeamercolor{title}{fg=darkblue}
\setbeamercolor{structure}{fg=darkblue}
\setbeamercolor{block title}{bg=darkblue, fg=white}
\setbeamercolor{block body}{bg=lightblue!30}

% ============================================================
% Title Information
% ============================================================
\title[CTMRG]{\textbf{Corner Transfer Matrix Renormalization Group}}
\subtitle{Efficient Contraction of 2D Classical Lattice Models}
\author{Your Name}
\institute{Your Institution}
\date{\today}

% ============================================================
% Custom Commands
% ============================================================
\newcommand{\highlight}[1]{\textcolor{accentred}{\textbf{#1}}}
\newcommand{\goodpoint}[1]{\textcolor{accentgreen}{\textbf{#1}}}

% ============================================================
% Document Begin
% ============================================================
\begin{document}

% ------------------------------------------------------------
% Title Slide
% ------------------------------------------------------------
\begin{frame}
    \titlepage
\end{frame}

% ------------------------------------------------------------
% Outline
% ------------------------------------------------------------
\begin{frame}{Outline}
    \tableofcontents
\end{frame}

% ============================================================
% PART 1: From 1D to 2D - The Exponential Wall
% ============================================================
\section{From 1D to 2D: The Exponential Wall}

% ------------------------------------------------------------
% Slide 1.1: 1D Ising Review
% ------------------------------------------------------------
\begin{frame}{Review: 1D Transfer Matrix (Your Pre-knowledge)}
    
    \begin{columns}
        \begin{column}{0.5\textwidth}
            \textbf{1D Ising Model:}
            \[
            H = -J \sum_i \sigma_i \sigma_{i+1}
            \]
            
            \vspace{0.5cm}
            
            \textbf{Transfer Matrix:}
            \[
            Z = \text{Tr}(T^N), \quad T \in \mathbb{R}^{2 \times 2}
            \]
            
            \[
            T = \begin{pmatrix} e^{\beta J} & e^{-\beta J} \\ e^{-\beta J} & e^{\beta J} \end{pmatrix}
            \]
        \end{column}
        
        \begin{column}{0.5\textwidth}
            % === FIGURE: 1D chain with transfer matrices ===
            \begin{center}
                \includegraphics[width=0.95\textwidth]{figures/fig_1d_transfer_chain.pdf}
            \end{center}
        \end{column}
    \end{columns}
    
    \vspace{0.3cm}
    \begin{block}{Key Point}
        $T$ is $2 \times 2$ $\Rightarrow$ \goodpoint{Exact diagonalization is trivial!}
    \end{block}
\end{frame}

% ------------------------------------------------------------
% Slide 1.2: When 1D Becomes Hard
% ------------------------------------------------------------
\begin{frame}{When Even 1D Has an Exponential Wall}
    
    \textbf{Recall:} For uniform 1D Ising, $Z = \text{Tr}(T^N)$ with $T \in \mathbb{R}^{2\times 2}$ is \goodpoint{exact}.
    
    \vspace{0.3cm}
    
    \begin{columns}
        \begin{column}{0.5\textwidth}
            \textbf{But what if...}
            \begin{itemize}
                \item Couplings $J_i$ are \highlight{site-dependent}?
                \item Random disorder: $J_i \sim \mathcal{N}(\bar{J}, \sigma)$?
                \item Open boundary conditions (no translation symmetry)?
            \end{itemize}
            
            \vspace{0.3cm}
            
            \textbf{Then:} Cannot diagonalize $T$ once!
            \[
            Z = \text{Tr}(T_1 \cdot T_2 \cdot T_3 \cdots T_N)
            \]
            Each $T_i$ is different $\Rightarrow$ no simple $\lambda^N$ formula.
        \end{column}
        
        \begin{column}{0.5\textwidth}
            % === FIGURE PLACEHOLDER ===
            % Concept: Chain with varying couplings
            %
            % ASCII draft:
            %   ○──J₁──○──J₂──○──J₃──○──J₄──○
            %      T₁     T₂     T₃     T₄
            %   
            %   Z = Tr(T₁ · T₂ · T₃ · T₄ · ...)
            %   
            %   "No single eigenvalue trick!"
            \begin{center}
                \fbox{\parbox{0.95\textwidth}{
                    \centering
                    \vspace{2cm}
                    \textit{[Figure: 1D chain with non-uniform couplings $J_i$]}
                    \vspace{2cm}
                }}
            \end{center}
            
            \begin{alertblock}{The Problem}
                For $N$ sites: need $O(N)$ matrix multiplications.
                
                Not exponential, but \highlight{no closed-form solution}.
            \end{alertblock}
        \end{column}
    \end{columns}
\end{frame}

% ------------------------------------------------------------
% Slide 1.3: 1D Solution - Transfer Matrix RG
% ------------------------------------------------------------
\begin{frame}{Solution: Transfer Matrix Renormalization Group (TMRG)}
    
    \begin{center}
        \textbf{Key Insight:} Not all configurations contribute equally to $Z$.
        
        \vspace{0.2cm}
        
        $\Rightarrow$ Keep only the \goodpoint{most relevant} ones!
    \end{center}
    
    \vspace{0.3cm}
    
    \begin{columns}
        \begin{column}{0.55\textwidth}
            % === FIGURE PLACEHOLDER ===
            % Concept: Left environment - local site - Right environment
            %
            % ASCII draft:
            %   Z = Tr([L] · T · T · [R])
            %         ↑              ↑
            %    Left env.      Right env.
            %   (compressed)    (compressed)
            %   
            %   L = sum over all left configs  → χ states
            %   R = sum over all right configs → χ states
            \begin{center}
                \fbox{\parbox{0.95\textwidth}{
                    \centering
                    \vspace{2.5cm}
                    \textit{[Figure: $Z \approx \text{Tr}(L \cdot T \cdot T \cdot R)$]}
                    \vspace{2.5cm}
                }}
            \end{center}
        \end{column}
        
        \begin{column}{0.45\textwidth}
            \textbf{TMRG Algorithm:}
            \begin{enumerate}
                \item \textbf{Grow:} $L' = L \cdot T$ 
                
                (add sites $\approx$ decimation)
                
                \item \textbf{Truncate:} SVD $\to$ keep $\chi$ largest
                
                (coarse-grain $\approx$ rescaling)
            \end{enumerate}
            
            \vspace{0.3cm}
            
            \begin{block}{RG Fixed Point}
                Iterate until $L^*, R^*$ converge.
                
                $\Rightarrow$ Thermodynamic limit!
            \end{block}
        \end{column}
    \end{columns}
    
    \vspace{0.2cm}
    
    \begin{center}
        \textit{(Detailed derivation in Appendix)}
    \end{center}
\end{frame}

% ============================================================
% PART 2: Going to 2D - The Network Problem
% ============================================================
\section{Going to 2D: A Fundamentally Harder Problem}

% ------------------------------------------------------------
% Slide 2.1: 2D is a Network!
% ------------------------------------------------------------
\begin{frame}{2D: The Transfer Matrix Becomes a \highlight{Network}}
    
    \begin{columns}
        \begin{column}{0.45\textwidth}
            \textbf{1D:} Ordered product of matrices
            \[
            Z = \text{Tr}(T_1 \cdot T_2 \cdots T_N)
            \]
            Contract left-to-right: $O(N)$.
            
            \vspace{0.5cm}
            
            \textbf{2D:} Each site couples to \highlight{4 neighbors}!
            \[
            T^{\sigma_i}_{\sigma_\text{up}, \sigma_\text{down}, \sigma_\text{left}, \sigma_\text{right}}
            \]
            
            This is a \highlight{rank-4 tensor}, not a matrix!
        \end{column}
        
        \begin{column}{0.55\textwidth}
            % === FIGURE PLACEHOLDER ===
            % Concept: 2D lattice showing each site has 4 bonds
            %
            % ASCII draft:
            %   σ_{i,j-1}
            %       │
            %   σ_{i-1,j}──[T]──σ_{i+1,j}
            %       │
            %   σ_{i,j+1}
            %
            %   Z = Σ_{all σ} Π_{sites} T^σ_{neighbors}
            %
            %   "No natural ordering for contraction!"
            \begin{center}
                \fbox{\parbox{0.95\textwidth}{
                    \centering
                    \vspace{2.5cm}
                    \textit{[Figure: 2D lattice --- each site is a rank-4 tensor]}
                    \vspace{2.5cm}
                }}
            \end{center}
        \end{column}
    \end{columns}
    
    \vspace{0.3cm}
    
    \begin{alertblock}{The Fundamental Problem}
        \highlight{No natural contraction order!} The 2D tensor network cannot be reduced to a simple trace of matrix products. \textbf{Exact contraction is \#P-hard!}
    \end{alertblock}
\end{frame}

% ------------------------------------------------------------
% Slide 2.2: One Approach - Row-to-Row (Brief)
% ------------------------------------------------------------
\begin{frame}{One Approach: Row-to-Row Transfer (DMRG-style)}
    
    \textbf{Idea:} Group one row of $L$ spins $\to$ treat as a ``super-spin'' with $2^L$ states.
    
    \vspace{0.3cm}
    
    \begin{columns}
        \begin{column}{0.5\textwidth}
            % === FIGURE PLACEHOLDER ===
            % Concept: Row grouped as super-spin
            %
            % ASCII draft:
            %   Row n:   ○──○──○──○──○  →  [σ_row] (2^L states)
            %            │  │  │  │  │
            %   Row n+1: ○──○──○──○──○  →  [σ'_row]
            %
            %   T_row ∈ R^{2^L × 2^L}
            \begin{center}
                \fbox{\parbox{0.95\textwidth}{
                    \centering
                    \vspace{1.8cm}
                    \textit{[Figure: Row $\to$ super-spin, $T_{\text{row}} \in \mathbb{R}^{2^L \times 2^L}$]}
                    \vspace{1.8cm}
                }}
            \end{center}
            
            Then $Z = \text{Tr}(T_{\text{row}}^M)$ looks like 1D!
        \end{column}
        
        \begin{column}{0.5\textwidth}
            \textbf{Apply DMRG/TMRG ideas:}
            \begin{itemize}
                \item Row config space $\to$ MPS
                \item Row-to-row transfer $\to$ MPO
                \item Truncate via SVD
            \end{itemize}
            
            \vspace{0.3cm}
            
            \begin{block}{Pros \& Cons}
                \textcolor{accentgreen}{\textbf{+}} Systematic, well-understood
                
                \textcolor{accentred}{\textbf{--}} Breaks 2D symmetry
                
                \textcolor{accentred}{\textbf{--}} Hard to generalize to other lattices
            \end{block}
            
            \vspace{0.2cm}
            
            \textit{(Details in Appendix)}
        \end{column}
    \end{columns}
\end{frame}

% ------------------------------------------------------------
% Slide 2.3: A More Natural 2D Perspective
% ------------------------------------------------------------
\begin{frame}{A More Natural Approach: Think in 2D!}
    
    \begin{center}
        \textbf{Question:} If we only want \highlight{local observables} $\langle \sigma_{i,j} \rangle$, 
        
        do we really need to contract the \textit{entire} infinite lattice?
    \end{center}
    
    \vspace{0.5cm}
    
    \begin{columns}
        \begin{column}{0.5\textwidth}
            % === FIGURE PLACEHOLDER ===
            % Concept: Local site surrounded by environment
            %
            % ASCII draft:
            %   ┌─────────────────┐
            %   │  Environment    │
            %   │    ┌───┐        │
            %   │    │ σ │ ← local│
            %   │    └───┘        │
            %   │                 │
            %   └─────────────────┘
            %
            %   "Environment encodes the rest of the lattice"
            \begin{center}
                \fbox{\parbox{0.95\textwidth}{
                    \centering
                    \vspace{2.2cm}
                    \textit{[Figure: Local site surrounded by 2D environment]}
                    \vspace{2.2cm}
                }}
            \end{center}
        \end{column}
        
        \begin{column}{0.5\textwidth}
            \textbf{Key Insight:}
            
            Decompose the infinite 2D environment into \highlight{geometrically natural} pieces:
            
            \vspace{0.3cm}
            
            \begin{itemize}
                \item \textbf{4 Corners} (quarter-planes)
                \item \textbf{4 Edges} (half-infinite strips)
            \end{itemize}
            
            \vspace{0.3cm}
            
            \begin{block}{Baxter's Insight (1968)}
                The \textbf{Corner Transfer Matrix} encodes a quarter of the infinite plane!
            \end{block}
        \end{column}
    \end{columns}
    
    \vspace{0.3cm}
    
    \begin{center}
        $\Rightarrow$ \textbf{CTMRG:} Apply RG to corners and edges!
    \end{center}
\end{frame}

% ============================================================
% PART 3: CTMRG for 2D
% ============================================================
\section{CTMRG: Extending to 2D}

% ------------------------------------------------------------
% Slide 3.1: 2D Environment Structure
% ------------------------------------------------------------
\begin{frame}{2D Analog: Four Corners + Four Edges}
    
    \textbf{Key Idea:} Divide infinite 2D lattice into \highlight{4 corners} + \highlight{4 edges}.
    
    \vspace{0.3cm}
    
    % === FIGURE PLACEHOLDER ===
    % Concept: The full 2D environment structure
    %
    % ASCII draft:
    %        C₄ ─── T₄ ─── C₁
    %         │             │
    %        T₃      ●     T₁
    %         │             │
    %        C₃ ─── T₂ ─── C₂
    %
    % C = corner matrices (χ × χ)
    % T = edge tensors (χ × d × χ)
    % ● = local tensor (center)
    %
    % Color code: corners in one color, edges in another
    \begin{center}
        \fbox{\parbox{0.55\textwidth}{
            \centering
            \vspace{3.5cm}
            \textit{[Figure: 2D environment structure --- 4 corners + 4 edges]}
            \vspace{3.5cm}
        }}
    \end{center}
    
    \vspace{0.3cm}
    
    \begin{columns}
        \begin{column}{0.5\textwidth}
            \textbf{Corners $C_1, C_2, C_3, C_4$:}
            \begin{itemize}
                \item Encode quadrant contributions
                \item Size: $\chi \times \chi$
            \end{itemize}
        \end{column}
        \begin{column}{0.5\textwidth}
            \textbf{Edges $T_1, T_2, T_3, T_4$:}
            \begin{itemize}
                \item Half-infinite row/column
                \item Size: $\chi \times d \times \chi$
            \end{itemize}
        \end{column}
    \end{columns}
\end{frame}

% ------------------------------------------------------------
% Slide 3.2: Local Tensor
% ------------------------------------------------------------
\begin{frame}{The Local Tensor: Building Block}
    
    \begin{columns}
        \begin{column}{0.5\textwidth}
            \textbf{Local Boltzmann weight tensor $a$:}
            
            \vspace{0.5cm}
            
            % === FIGURE PLACEHOLDER ===
            % Concept: 4-leg tensor for one site
            %
            % ASCII draft:
            %              up
            %               │
            %    left ──── [a] ──── right
            %               │
            %             down
            %
            % Each leg connects to a neighbor
            \begin{center}
                \fbox{\parbox{0.8\textwidth}{
                    \centering
                    \vspace{2cm}
                    \textit{[Figure: Local tensor $a$ with 4 legs]}
                    \vspace{2cm}
                }}
            \end{center}
        \end{column}
        
        \begin{column}{0.5\textwidth}
            \textbf{For 2D Ising:}
            \[
            a_{u,d,l,r} = \sum_{\sigma} W_{\sigma,u} W_{\sigma,d} W_{\sigma,l} W_{\sigma,r}
            \]
            
            where
            \[
            W_{\sigma, \sigma'} = e^{\frac{\beta J}{2} \sigma \sigma'}
            \]
            
            \vspace{0.3cm}
            
            \begin{block}{Physical Meaning}
                $a$ encodes the Boltzmann weight at one site, with bond weights split symmetrically.
            \end{block}
        \end{column}
    \end{columns}
\end{frame}

% ------------------------------------------------------------
% Slide 3.3: CTMRG Grow Step
% ------------------------------------------------------------
\begin{frame}{CTMRG Iteration: Step 1 --- Grow (Absorb Row/Column)}
    
    \textbf{Absorption:} Add one row \textbf{and} one column to expand the environment.
    
    \vspace{0.3cm}
    
    % === FIGURE PLACEHOLDER ===
    % Concept: Corner absorption step
    %
    % ASCII draft (showing one direction):
    %   Before:           After absorption:
    %   
    %   C₁ ─── T₁         C₁'────────
    %    │      │    →     │         │
    %   T₄ ─── a          T₄'       T₁'
    %                      │         │
    %                      └─── a ───┘
    %
    % Show how corner "eats" the local tensor
    % Bond dimension grows: χ → χ·d
    \begin{center}
        \fbox{\parbox{0.8\textwidth}{
            \centering
            \vspace{3cm}
            \textit{[Figure: Absorption step --- corner grows by absorbing $a$ and edges]}
            \vspace{3cm}
        }}
    \end{center}
    
    \vspace{0.3cm}
    
    \begin{alertblock}{Bond Dimension Growth}
        After absorption: $C'$ has size $(\chi \cdot d) \times (\chi \cdot d)$ --- \highlight{too large!}
    \end{alertblock}
\end{frame}

% ------------------------------------------------------------
% Slide 3.4: CTMRG Truncate Step
% ------------------------------------------------------------
\begin{frame}{CTMRG Iteration: Step 2 --- Truncate (The Key Difference!)}
    
    \textbf{Truncation in 2D:} Use the \highlight{full environment} to determine projectors.
    
    \vspace{0.3cm}
    
    % === FIGURE PLACEHOLDER ===
    % Concept: How to compute projector in 2D
    %
    % ASCII draft:
    %   Build "half-system" density matrix:
    %   
    %        C₄' ─── T₄' ─── C₁'
    %         │               │
    %        T₃'             T₁'
    %         │               │
    %        ─────── ρ ───────
    %                ↓
    %         SVD/Eigendecomposition
    %                ↓
    %         Projector P (keep χ largest)
    %
    % This is different from 1D: need to consider 2D geometry!
    \begin{center}
        \fbox{\parbox{0.75\textwidth}{
            \centering
            \vspace{2.5cm}
            \textit{[Figure: Building density matrix from environment for truncation]}
            \vspace{2.5cm}
        }}
    \end{center}
    
    \vspace{0.3cm}
    
    \begin{columns}
        \begin{column}{0.55\textwidth}
            \textbf{Compute projector $P$:}
            \begin{enumerate}
                \item Contract half-environment $\to \rho$
                \item SVD: $\rho = U \Sigma V^\dagger$
                \item Keep $\chi$ largest: $P = U_{:\chi}$
            \end{enumerate}
        \end{column}
        \begin{column}{0.45\textwidth}
            \textbf{Apply truncation:}
            \[
            C_{\text{new}} = P^\dagger C' P
            \]
            \[
            T_{\text{new}} = P^\dagger T' P
            \]
        \end{column}
    \end{columns}
\end{frame}

% ------------------------------------------------------------
% Slide 3.5: CTMRG Full Algorithm
% ------------------------------------------------------------
\begin{frame}{CTMRG: Complete Algorithm}
    
    % === FIGURE PLACEHOLDER ===
    % Concept: Flowchart of CTMRG algorithm
    %
    % ASCII draft:
    %   ┌─────────────────────────────────────┐
    %   │ Initialize: C₁, C₂, C₃, C₄, T₁-T₄  │
    %   └───────────────┬─────────────────────┘
    %                   ↓
    %   ┌─────────────────────────────────────┐
    %   │ 1. Absorb: grow all corners & edges │◄───┐
    %   └───────────────┬─────────────────────┘    │
    %                   ↓                          │
    %   ┌─────────────────────────────────────┐    │
    %   │ 2. Compute projectors P from ρ      │    │
    %   └───────────────┬─────────────────────┘    │
    %                   ↓                          │
    %   ┌─────────────────────────────────────┐    │
    %   │ 3. Truncate: C_new = P† C' P        │    │
    %   └───────────────┬─────────────────────┘    │
    %                   ↓                          │
    %   ┌─────────────────────────────────────┐    │
    %   │ 4. Converged?  ──No──────────────────────┘
    %   └───────────────┬─────────────────────┘
    %                   │ Yes
    %                   ↓
    %   ┌─────────────────────────────────────┐
    %   │ Output: Fixed-point environment     │
    %   └─────────────────────────────────────┘
    \begin{center}
        \fbox{\parbox{0.7\textwidth}{
            \centering
            \vspace{4cm}
            \textit{[Figure: CTMRG algorithm flowchart]}
            \vspace{4cm}
        }}
    \end{center}
\end{frame}

% ------------------------------------------------------------
% Slide 3.6: Computing Observables
% ------------------------------------------------------------
加个local极化,能量……
\begin{frame}{Computing Observables with CTMRG}
    
    \textbf{Partition function:}
    \[
    Z \propto \text{Tr}(C_1 T_1 C_2 T_2 C_3 T_3 C_4 T_4)
    \]
    
    \vspace{0.3cm}
    
    % === FIGURE PLACEHOLDER ===
    % Concept: How to compute local magnetization
    %
    % ASCII draft:
    %        C₄ ─── T₄ ─── C₁
    %         │             │
    %        T₃      σ     T₁    ← insert σ at center
    %         │             │
    %        C₃ ─── T₂ ─── C₂
    %
    % Contract everything to get <σ>
    \begin{center}
        \fbox{\parbox{0.6\textwidth}{
            \centering
            \vspace{2.5cm}
            \textit{[Figure: Computing $\langle \sigma \rangle$ by inserting operator]}
            \vspace{2.5cm}
        }}
    \end{center}
    
    \vspace{0.3cm}
    
    \textbf{Local magnetization:}
    \[
    \langle \sigma_{\text{center}} \rangle = \frac{\text{Tr}(C_1 T_1 C_2 T_2 C_3 T_3 C_4 T_4 \cdot \sigma)}{Z}
    \]
\end{frame}

% ============================================================
% PART 4: Results
% ============================================================
\section{Results}

% ------------------------------------------------------------
% Slide 4.1: Results (Placeholder)
% ------------------------------------------------------------
\begin{frame}{Numerical Results: 2D Ising Model}
    
    % === RESULTS PLACEHOLDER ===
    % To be filled with actual numerical results
    % Suggestions:
    % - Magnetization vs temperature curve
    % - Comparison with exact Onsager solution
    % - Convergence with bond dimension χ
    % - Critical exponents
    
    \begin{center}
        \fbox{\parbox{0.85\textwidth}{
            \centering
            \vspace{5cm}
            \textit{[Results to be added: magnetization, critical behavior, etc.]}
            \vspace{5cm}
        }}
    \end{center}
    
\end{frame}

% ============================================================
% PART 5: Summary
% ============================================================
\section{Summary}

% ------------------------------------------------------------
% Slide 5.1: Core Steps
% ------------------------------------------------------------
\begin{frame}{Summary: The Core of CTMRG}
    
    \begin{block}{CTMRG in Three Steps}
        \begin{enumerate}
            \item \textbf{Decompose:} Infinite 2D lattice $\to$ 4 corners + 4 edges
            \item \textbf{Grow:} Absorb local tensors (add row/column)
            \item \textbf{Truncate:} SVD-based RG to keep $\chi$ most relevant states
        \end{enumerate}
    \end{block}
    
    \vspace{0.5cm}
    
    % === FIGURE PLACEHOLDER ===
    % Concept: Visual summary of the three steps
    %
    % ASCII draft:
    %   [Decompose]      [Grow]          [Truncate]
    %   4 corners    →   absorb a    →   SVD + keep χ
    %   + 4 edges        (larger)        (back to χ)
    %
    % Show as three-panel diagram
    \begin{center}
        \fbox{\parbox{0.85\textwidth}{
            \centering
            \vspace{2.5cm}
            \textit{[Figure: Visual summary --- Decompose $\to$ Grow $\to$ Truncate]}
            \vspace{2.5cm}
        }}
    \end{center}
\end{frame}

% ------------------------------------------------------------
% Slide 5.2: DMRG vs CTMRG
% ------------------------------------------------------------
直接放附录!直接放附录!直接放附录!
\begin{frame}{Comparison: iDMRG vs CTMRG}
    
    \begin{table}
        \centering
        \renewcommand{\arraystretch}{1.3}
        \begin{tabular}{l|c|c}
            \toprule
            \textbf{Aspect} & \textbf{iDMRG (1D)} & \textbf{CTMRG (2D)} \\
            \midrule
            Dimension & 1D chain & 2D square lattice \\
            \midrule
            Environment & Left + Right & 4 Corners + 4 Edges \\
            \midrule
            Grow step & Add site pair & Add row + column \\
            \midrule
            Truncation & SVD on center bond & SVD on corner boundaries \\
            \midrule
            Bond dimension & $\chi$ (MPS) & $\chi$ (environment) \\
            \midrule
            Fixed point & $L^*, R^*$ & $C_i^*, T_i^*$ \\
            \midrule
            Computational cost & $O(\chi^3)$ & $O(\chi^6)$ or $O(\chi^5)$ \\
            \bottomrule
        \end{tabular}
    \end{table}
    
    \vspace{0.3cm}
    
    \begin{block}{Take-Home Message}
        \goodpoint{Same RG philosophy}, different geometric structure!
    \end{block}
\end{frame}

% ============================================================
% Thank You Slide
% ============================================================
\begin{frame}
    \begin{center}
        \Huge \textbf{Thank You!}
        
        \vspace{1cm}
        
        \Large Questions?
    \end{center}
\end{frame}

% ============================================================
% APPENDIX
% ============================================================
\appendix
\section*{Appendix}

% ============================================================
% Appendix B: 1D Transfer Matrix RG (TMRG) - Detailed
% ============================================================

% ------------------------------------------------------------
% Appendix B.1: TMRG Setup
% ------------------------------------------------------------
\begin{frame}{Appendix: TMRG Setup in Detail}
    
    \textbf{Goal:} Compute $Z = \text{Tr}(T_1 \cdot T_2 \cdots T_N)$ for large $N$.
    
    \vspace{0.3cm}
    
    % === FIGURE PLACEHOLDER ===
    \begin{center}
        \fbox{\parbox{0.8\textwidth}{
            \centering
            \vspace{2cm}
            \textit{[Figure: Left environment $L$, local site $T$, right environment $R$]}
            \vspace{2cm}
        }}
    \end{center}
    
    \vspace{0.3cm}
    
    \begin{columns}
        \begin{column}{0.5\textwidth}
            \textbf{Left Environment $L$:}
            \[
            L_{\sigma} = \sum_{\sigma_1, \ldots, \sigma_{k-1}} T_1 \cdot T_2 \cdots T_{k-1}
            \]
            Encodes all configs to the left of site $k$.
        \end{column}
        \begin{column}{0.5\textwidth}
            \textbf{Right Environment $R$:}
            \[
            R_{\sigma} = \sum_{\sigma_{k+1}, \ldots, \sigma_N} T_{k+1} \cdots T_N
            \]
            Encodes all configs to the right of site $k$.
        \end{column}
    \end{columns}
\end{frame}

% ------------------------------------------------------------
% Appendix B.2: TMRG Grow Step
% ------------------------------------------------------------
\begin{frame}{Appendix: TMRG Step 1 --- Grow (Absorption)}
    
    \textbf{Absorption:} Add one more site to the environment.
    
    \vspace{0.3cm}
    
    \begin{center}
        \fbox{\parbox{0.8\textwidth}{
            \centering
            \vspace{2cm}
            \textit{[Figure: $L' = L \cdot T_k$, growing the left environment]}
            \vspace{2cm}
        }}
    \end{center}
    
    \vspace{0.3cm}
    
    \textbf{Mathematically:}
    \[
    L'_{\sigma_k} = \sum_{\sigma_{k-1}} L_{\sigma_{k-1}} \cdot (T_k)_{\sigma_{k-1}, \sigma_k}
    \]
    
    \begin{alertblock}{Problem}
        If $L$ has dimension $\chi$, after absorption $L'$ has dimension $\chi \cdot d$.
        
        After $N$ absorptions: dimension $\sim d^N$ --- \highlight{exponential}!
    \end{alertblock}
\end{frame}

% ------------------------------------------------------------
% Appendix B.3: TMRG Truncate Step
% ------------------------------------------------------------
\begin{frame}{Appendix: TMRG Step 2 --- Truncate (SVD)}
    
    \textbf{Idea:} Compress $L'$ back to dimension $\chi$ using SVD.
    
    \vspace{0.3cm}
    
    \begin{columns}
        \begin{column}{0.5\textwidth}
            \textbf{Form the ``density matrix'':}
            \[
            \rho = L' \cdot R'^T
            \]
            
            \textbf{SVD:}
            \[
            \rho = U \Sigma V^\dagger
            \]
            
            \textbf{Truncate:}
            \[
            P = U_{:, 1:\chi}
            \]
            
            \textbf{New environment:}
            \[
            L_{\text{new}} = P^\dagger L'
            \]
        \end{column}
        
        \begin{column}{0.5\textwidth}
            \begin{center}
                \fbox{\parbox{0.95\textwidth}{
                    \centering
                    \vspace{2cm}
                    \textit{[Figure: SVD spectrum, keep $\chi$ largest]}
                    \vspace{2cm}
                }}
            \end{center}
            
            \begin{block}{Eckart-Young Theorem}
                SVD gives the \textbf{optimal} rank-$\chi$ approximation in Frobenius norm.
            \end{block}
        \end{column}
    \end{columns}
\end{frame}

% ------------------------------------------------------------
% Appendix B.4: TMRG Convergence
% ------------------------------------------------------------
\begin{frame}{Appendix: TMRG Convergence and Observables}
    
    \textbf{Iterate:} Grow $\to$ Truncate $\to$ Grow $\to$ Truncate $\to \cdots$
    
    \vspace{0.3cm}
    
    \begin{columns}
        \begin{column}{0.5\textwidth}
            \textbf{Convergence criterion:}
            
            Fixed point: $L^*, R^*$ such that
            \[
            L^* \xrightarrow{\text{grow+truncate}} L^*
            \]
            
            \vspace{0.3cm}
            
            \textbf{Free energy:}
            \[
            f = -k_B T \lim_{N\to\infty} \frac{1}{N} \ln Z
            \]
            \[
            = -k_B T \ln \sigma_1^*
            \]
        \end{column}
        
        \begin{column}{0.5\textwidth}
            \textbf{Local observables:}
            \[
            \langle \sigma_k \rangle = \frac{L^* \cdot \sigma_k \cdot R^*}{L^* \cdot R^*}
            \]
            
            \textbf{Correlation functions:}
            \[
            \langle \sigma_i \sigma_j \rangle = \frac{L^* \cdot \sigma_i \cdot T^{|i-j|} \cdot \sigma_j \cdot R^*}{Z}
            \]
            
            \vspace{0.3cm}
            
            \begin{block}{Physical Meaning}
                $L^*, R^*$ encode the \textbf{thermodynamic limit}.
            \end{block}
        \end{column}
    \end{columns}
\end{frame}

% ============================================================
% Appendix C: 2D Row-to-Row DMRG - Detailed
% ============================================================

% ------------------------------------------------------------
% Appendix C.1: Row as Super-Spin
% ------------------------------------------------------------
\begin{frame}{Appendix: 2D Row-to-Row --- Row as Super-Spin}
    
    \textbf{Idea:} Treat one row of $L$ spins as a single ``super-spin''.
    
    \vspace{0.3cm}
    
    \begin{columns}
        \begin{column}{0.5\textwidth}
            \begin{center}
                \fbox{\parbox{0.95\textwidth}{
                    \centering
                    \vspace{2cm}
                    \textit{[Figure: Row $\to$ super-spin with $2^L$ states]}
                    \vspace{2cm}
                }}
            \end{center}
        \end{column}
        
        \begin{column}{0.5\textwidth}
            \textbf{Row configuration:}
            \[
            \vec{\sigma} = (\sigma_1, \sigma_2, \ldots, \sigma_L)
            \]
            
            Total states: $2^L$.
            
            \vspace{0.3cm}
            
            \textbf{Row-to-row transfer:}
            \[
            (T_{\text{row}})_{\vec{\sigma}, \vec{\sigma}'} = \prod_{i=1}^L e^{\beta J \sigma_i \sigma'_i} \prod_{i=1}^{L-1} e^{\beta J \sigma_i \sigma_{i+1}}
            \]
            
            $T_{\text{row}} \in \mathbb{R}^{2^L \times 2^L}$.
        \end{column}
    \end{columns}
    
    \vspace{0.3cm}
    
    \begin{block}{Key Point}
        Now $Z = \text{Tr}(T_{\text{row}}^M)$ looks like 1D problem with $2^L$-dimensional transfer matrix!
    \end{block}
\end{frame}

% ------------------------------------------------------------
% Appendix C.2: MPO Decomposition
% ------------------------------------------------------------
\begin{frame}{Appendix: Row Transfer Matrix = MPO}
    
    \textbf{Key insight:} $T_{\text{row}}$ has a \highlight{tensor network} structure!
    
    \vspace{0.3cm}
    
    \begin{columns}
        \begin{column}{0.55\textwidth}
            \textbf{Decompose} $T_{\text{row}}$ into local tensors:
            \[
            T_{\vec{\sigma}, \vec{\sigma}'} = \sum_{\alpha_1, \ldots} W^{\sigma_1 \sigma'_1}_{\alpha_0 \alpha_1} W^{\sigma_2 \sigma'_2}_{\alpha_1 \alpha_2} \cdots
            \]
            
            \begin{center}
                \fbox{\parbox{0.95\textwidth}{
                    \centering
                    \vspace{1.5cm}
                    \textit{[Figure: MPO structure of $T_{\text{row}}$]}
                    \vspace{1.5cm}
                }}
            \end{center}
        \end{column}
        
        \begin{column}{0.45\textwidth}
            \textbf{Local tensor $W$:}
            \[
            W^{\sigma \sigma'}_{\alpha \beta} = \text{local Boltzmann weight}
            \]
            
            \begin{itemize}
                \item $\sigma, \sigma'$: spins in rows $n$, $n+1$
                \item $\alpha, \beta$: auxiliary (horizontal bonds)
            \end{itemize}
            
            \vspace{0.2cm}
            
            \begin{block}{No Quantum!}
                ``MPO'' is just a factorization of the classical transfer matrix.
            \end{block}
        \end{column}
    \end{columns}
\end{frame}

% ------------------------------------------------------------
% Appendix C.3: MPS for Row Configuration
% ------------------------------------------------------------
\begin{frame}{Appendix: Boundary = MPS}
    
    \textbf{MPS:} Efficient representation of row configuration space.
    
    \vspace{0.3cm}
    
    \begin{columns}
        \begin{column}{0.5\textwidth}
            \textbf{Full vector:}
            \[
            |R\rangle = \sum_{\vec{\sigma}} R_{\vec{\sigma}} |\vec{\sigma}\rangle
            \]
            has $2^L$ components.
            
            \vspace{0.3cm}
            
            \textbf{MPS compression:}
            \[
            R_{\vec{\sigma}} = A^{\sigma_1} A^{\sigma_2} \cdots A^{\sigma_L}
            \]
            
            Only $L \cdot \chi^2 \cdot 2$ parameters!
        \end{column}
        
        \begin{column}{0.5\textwidth}
            \begin{center}
                \fbox{\parbox{0.95\textwidth}{
                    \centering
                    \vspace{1.8cm}
                    \textit{[Figure: MPS tensor network]}
                    \vspace{1.8cm}
                }}
            \end{center}
            
            \begin{block}{Physical Meaning}
                MPS compresses $2^L$ row configs into $\chi$ effective states with \textbf{limited entanglement}.
            \end{block}
        \end{column}
    \end{columns}
\end{frame}

% ------------------------------------------------------------
% Appendix C.4: Row-to-Row DMRG Algorithm
% ------------------------------------------------------------
\begin{frame}{Appendix: Row-to-Row DMRG Algorithm}
    
    \textbf{Combine:} MPS (boundary) + MPO (transfer) + SVD (truncation).
    
    \vspace{0.3cm}
    
    \begin{enumerate}
        \item Initialize MPS $|L\rangle$, $|R\rangle$ for boundaries
        
        \item \textbf{Grow:} Apply MPO $T_{\text{row}}$ to boundaries
        \[
        |L'\rangle = T_{\text{row}} |L\rangle
        \]
        
        \item \textbf{Truncate:} SVD to compress bond dimension back to $\chi$
        
        \item Iterate until convergence to fixed point $|L^*\rangle$, $|R^*\rangle$
    \end{enumerate}
    
    \vspace{0.3cm}
    
    \begin{columns}
        \begin{column}{0.5\textwidth}
            \begin{block}{Pros}
                \begin{itemize}
                    \item Systematic, well-understood
                    \item Controlled approximation
                \end{itemize}
            \end{block}
        \end{column}
        \begin{column}{0.5\textwidth}
            \begin{alertblock}{Cons}
                \begin{itemize}
                    \item Breaks 2D rotational symmetry
                    \item Hard to generalize to other lattices
                \end{itemize}
            \end{alertblock}
        \end{column}
    \end{columns}
\end{frame}

% ============================================================
% Appendix A: Additional Topics
% ============================================================

% ------------------------------------------------------------
% Appendix A.1: No Sign Problem
% ------------------------------------------------------------
\begin{frame}{Appendix: No Monte Carlo Sign Problem}
    
    \textbf{Classical models:} All Boltzmann weights are \highlight{positive}!
    
    \[
    W = e^{-\beta H} > 0 \quad \text{always}
    \]
    
    \vspace{0.3cm}
    
    % === FIGURE PLACEHOLDER ===
    % Concept: Comparison of classical vs quantum MC
    %
    % ASCII draft:
    %   Classical:  W > 0  →  No sign problem  →  MC works!
    %   Quantum:    W can be < 0 (fermions, frustration)
    %                      →  Sign problem  →  MC fails
    \begin{center}
        \fbox{\parbox{0.7\textwidth}{
            \centering
            \vspace{2cm}
            \textit{[Figure: Classical (positive) vs Quantum (sign problem)]}
            \vspace{2cm}
        }}
    \end{center}
    
    \vspace{0.3cm}
    
    \begin{block}{Advantage of Tensor Network Methods}
        \begin{itemize}
            \item No statistical sampling $\Rightarrow$ no sign problem issue
            \item Deterministic algorithm with controlled error (via $\chi$)
            \item Works equally well for frustrated/quantum systems
        \end{itemize}
    \end{block}
\end{frame}

% ------------------------------------------------------------
% Appendix A.2: Why Not 3D?
% ------------------------------------------------------------
\begin{frame}{Appendix: Why is 3D Difficult?}
    
    \textbf{In 2D:} Environment tensors are \highlight{1D objects} (edges).
    
    \textbf{In 3D:} Environment would be \highlight{2D surfaces} --- back to exponential!
    
    \vspace{0.3cm}
    
    % === FIGURE PLACEHOLDER ===
    % Concept: 2D boundary in 3D vs 1D boundary in 2D
    %
    % ASCII draft:
    %   2D system:  boundary is 1D line (manageable with χ)
    %   3D system:  boundary is 2D surface (exponential again!)
    %
    % Show schematic of 3D cube with 2D boundary
    \begin{center}
        \fbox{\parbox{0.75\textwidth}{
            \centering
            \vspace{2.5cm}
            \textit{[Figure: 2D boundary problem in 3D systems]}
            \vspace{2.5cm}
        }}
    \end{center}
    
    \vspace{0.3cm}
    
    \begin{alertblock}{The 3D Challenge}
        No simple analog of CTMRG for 3D --- active research area!
        
        Possible approaches: Higher-order tensor RG, Monte Carlo + TN hybrids
    \end{alertblock}
\end{frame}

% ------------------------------------------------------------
% Appendix A.3: Extensions
% ------------------------------------------------------------
\begin{frame}{Appendix: Extensions and Generalizations}
    
    \textbf{Beyond square lattice Ising:}
    
    \vspace{0.3cm}
    
    \begin{columns}
        \begin{column}{0.5\textwidth}
            \textbf{Different lattices:}
            \begin{itemize}
                \item Honeycomb
                \item Triangular
                \item Kagome
            \end{itemize}
            
            \vspace{0.3cm}
            
            \textbf{Different models:}
            \begin{itemize}
                \item Potts model
                \item Clock model
                \item Vertex models
            \end{itemize}
        \end{column}
        
        \begin{column}{0.5\textwidth}
            \textbf{Quantum systems (via iPEPS):}
            \begin{itemize}
                \item 2D Heisenberg model
                \item Frustrated magnets
                \item Topological phases
            \end{itemize}
            
            \vspace{0.3cm}
            
            \textbf{Improvements:}
            \begin{itemize}
                \item Directional CTMRG
                \item Full-update vs simple-update
                \item Gradient optimization
            \end{itemize}
        \end{column}
    \end{columns}
\end{frame}

% ------------------------------------------------------------
% Appendix A.4: Computational Complexity
% ------------------------------------------------------------
\begin{frame}{Appendix: Computational Complexity}
    
    \textbf{CTMRG scaling:}
    
    \begin{table}
        \centering
        \begin{tabular}{l|c|c}
            \toprule
            \textbf{Operation} & \textbf{Cost} & \textbf{Bottleneck?} \\
            \midrule
            Corner absorption & $O(\chi^4 d^2)$ & \\
            Edge absorption & $O(\chi^3 d^2)$ & \\
            Build density matrix & $O(\chi^4)$ & \\
            SVD for projector & $O(\chi^3 d^3)$ & \checkmark \\
            Apply truncation & $O(\chi^3 d)$ & \\
            \midrule
            \textbf{Total per iteration} & $O(\chi^3 d^3)$ to $O(\chi^6)$ & \\
            \bottomrule
        \end{tabular}
    \end{table}
    
    \vspace{0.3cm}
    
    \begin{block}{Comparison}
        \begin{itemize}
            \item iDMRG (1D): $O(\chi^3)$ per sweep
            \item CTMRG (2D): $O(\chi^5)$ -- $O(\chi^6)$ per iteration
            \item More expensive, but still \goodpoint{polynomial} in $\chi$!
        \end{itemize}
    \end{block}
\end{frame}

% ------------------------------------------------------------
% Appendix A.5: Key References
% ------------------------------------------------------------
\begin{frame}{Appendix: Key References}
    
    \textbf{Original works:}
    \begin{itemize}
        \item R. J. Baxter, \textit{J. Math. Phys.} \textbf{9}, 650 (1968) --- Corner transfer matrices
        \item R. J. Baxter, \textit{J. Stat. Phys.} \textbf{19}, 461 (1978) --- CTM method
    \end{itemize}
    
    \vspace{0.3cm}
    
    \textbf{Modern CTMRG:}
    \begin{itemize}
        \item T. Nishino \& K. Okunishi, \textit{J. Phys. Soc. Jpn.} \textbf{65}, 891 (1996)
        \item T. Nishino \& K. Okunishi, \textit{J. Phys. Soc. Jpn.} \textbf{66}, 3040 (1997)
    \end{itemize}
    
    \vspace{0.3cm}
    
    \textbf{CTMRG for iPEPS:}
    \begin{itemize}
        \item R. Orús \& G. Vidal, \textit{Phys. Rev. B} \textbf{80}, 094403 (2009)
        \item P. Corboz et al., \textit{Phys. Rev. B} \textbf{84}, 041108(R) (2011)
    \end{itemize}
    
    \vspace{0.3cm}
    
    \textbf{Reviews:}
    \begin{itemize}
        \item R. Orús, \textit{Ann. Phys.} \textbf{349}, 117 (2014) --- Tensor networks review
    \end{itemize}
\end{frame}

\end{document}
