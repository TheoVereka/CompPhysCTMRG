% !TEX program = xelatex
\documentclass[aspectratio=169, 11pt]{beamer}

% ============================================================
% Packages
% ============================================================
\usepackage{amsmath, amssymb, amsfonts}
\usepackage{graphicx}
\usepackage{tikz}
\usetikzlibrary{shapes, arrows, positioning, calc, decorations.pathreplacing}
\usepackage{booktabs}
\usepackage{bm}
\usepackage{physics}
\usepackage{xcolor}
\usepackage{hyperref}

% ============================================================
% Theme and Colors
% ============================================================
\usetheme{Madrid}
\usecolortheme{seahorse}

\definecolor{darkblue}{RGB}{0, 51, 102}
\definecolor{lightblue}{RGB}{173, 216, 230}
\definecolor{accentred}{RGB}{178, 34, 34}
\definecolor{accentgreen}{RGB}{34, 139, 34}

\setbeamercolor{frametitle}{fg=darkblue}
\setbeamercolor{title}{fg=darkblue}
\setbeamercolor{structure}{fg=darkblue}
\setbeamercolor{block title}{bg=darkblue, fg=white}
\setbeamercolor{block body}{bg=lightblue!30}

% ============================================================
% Title Information
% ============================================================
\title[CTMRG]{\textbf{Corner Transfer Matrix Renormalization Group}}
\subtitle{Efficient Contraction of 2D Classical Lattice Models}
\author{Your Name}
\institute{Your Institution}
\date{\today}

% ============================================================
% Custom Commands
% ============================================================
\newcommand{\highlight}[1]{\textcolor{accentred}{\textbf{#1}}}
\newcommand{\goodpoint}[1]{\textcolor{accentgreen}{\textbf{#1}}}

% ============================================================
% Document Begin
% ============================================================
\begin{document}

% ------------------------------------------------------------
% Title Slide
% ------------------------------------------------------------
\begin{frame}
    \titlepage
\end{frame}

% ------------------------------------------------------------
% Outline
% ------------------------------------------------------------
\begin{frame}{Outline}
    \tableofcontents
\end{frame}

% ============================================================
% PART 1: From 1D to 2D - The Exponential Wall
% ============================================================
\section{From 1D to 2D: The Exponential Wall}

% ------------------------------------------------------------
% Slide 1.1: 1D Ising Review
% ------------------------------------------------------------
\begin{frame}{Review: 1D Transfer Matrix (Your Pre-knowledge)}
    
    \begin{columns}
        \begin{column}{0.5\textwidth}
            \textbf{1D Ising Model:}
            \[
            H = -J \sum_i \sigma_i \sigma_{i+1}
            \]
            
            \vspace{0.5cm}
            
            \textbf{Transfer Matrix:}
            \[
            Z = \text{Tr}(T^N), \quad T \in \mathbb{R}^{2 \times 2}
            \]
            
            \[
            T = \begin{pmatrix} e^{\beta J} & e^{-\beta J} \\ e^{-\beta J} & e^{\beta J} \end{pmatrix}
            \]
        \end{column}
        
        \begin{column}{0.5\textwidth}
            % === FIGURE PLACEHOLDER ===
            % Concept: 1D chain with transfer matrices connecting sites
            % 
            % ASCII draft:
            %   σ₁ ──[T]── σ₂ ──[T]── σ₃ ──[T]── σ₄ ──[T]── ...
            %
            % Show T as a small box/tensor between each pair
            % Emphasize: T is only 2×2, very manageable!
            \begin{center}
                \fbox{\parbox{0.9\textwidth}{
                    \centering
                    \vspace{2cm}
                    \textit{[Figure: 1D chain with transfer matrices]}
                    \vspace{2cm}
                }}
            \end{center}
        \end{column}
    \end{columns}
    
    \vspace{0.3cm}
    \begin{block}{Key Point}
        $T$ is $2 \times 2$ $\Rightarrow$ \goodpoint{Exact diagonalization is trivial!}
    \end{block}
\end{frame}

% ------------------------------------------------------------
% Slide 1.2: Going to 2D
% ------------------------------------------------------------
\begin{frame}{2D Square Lattice: Row-to-Row Transfer}
    
    \begin{columns}
        \begin{column}{0.45\textwidth}
            \textbf{2D Ising Model:}
            \[
            H = -J \sum_{\langle i,j \rangle} \sigma_i \sigma_j
            \]
            
            \vspace{0.3cm}
            
            \textbf{Row-to-Row Transfer:}
            \[
            Z = \text{Tr}(T_{\text{row}}^{M})
            \]
            
            \vspace{0.3cm}
            
            Row has $L$ spins $\Rightarrow$
            \[
            T_{\text{row}} \in \mathbb{R}^{2^L \times 2^L}
            \]
        \end{column}
        
        \begin{column}{0.55\textwidth}
            % === FIGURE PLACEHOLDER ===
            % Concept: 2D lattice with row-to-row transfer
            %
            % ASCII draft:
            %   Row n:   ○──○──○──○──○  (L spins = one "super-spin")
            %            │  │  │  │  │
            %          [====T_row====]  (transfer slab)
            %            │  │  │  │  │
            %   Row n+1: ○──○──○──○──○
            %
            % Emphasize the "slab" nature of T_row
            \begin{center}
                \fbox{\parbox{0.95\textwidth}{
                    \centering
                    \vspace{2.5cm}
                    \textit{[Figure: 2D lattice with row-to-row transfer matrix]}
                    \vspace{2.5cm}
                }}
            \end{center}
        \end{column}
    \end{columns}
\end{frame}

% ------------------------------------------------------------
% Slide 1.3: The Exponential Wall
% ------------------------------------------------------------
\begin{frame}{The Exponential Wall: A Dimensional Curse}
    
    \begin{center}
        % === FIGURE PLACEHOLDER ===
        % Concept: Exponential growth visualization
        %
        % ASCII draft: Bar chart or exponential curve
        %   L=10:  2^10 = 1,024         (small bar)
        %   L=20:  2^20 ≈ 10^6          (medium bar)
        %   L=50:  2^50 ≈ 10^15         (huge bar, off-scale)
        %   L=100: 2^100 ≈ 10^30        (impossibly large)
        %
        % Could also show: memory/time requirements
        \fbox{\parbox{0.7\textwidth}{
            \centering
            \vspace{2cm}
            \textit{[Figure: Exponential growth of $2^L$ with system size]}
            \vspace{2cm}
        }}
    \end{center}
    
    \vspace{0.3cm}
    
    \begin{columns}
        \begin{column}{0.5\textwidth}
            \begin{table}
                \centering
                \begin{tabular}{c|c}
                    \toprule
                    $L$ & Matrix Size $2^L$ \\
                    \midrule
                    10 & $\sim 10^3$ \\
                    20 & $\sim 10^6$ \\
                    50 & $\sim 10^{15}$ \\
                    100 & $\sim 10^{30}$ \\
                    \bottomrule
                \end{tabular}
            \end{table}
        \end{column}
        
        \begin{column}{0.5\textwidth}
            \begin{alertblock}{The Problem}
                \highlight{Exponential growth} is intrinsic to \textbf{higher dimensions}, not specific to row-to-row vs corner methods!
            \end{alertblock}
        \end{column}
    \end{columns}
\end{frame}

% ------------------------------------------------------------
% Slide 1.4: Need for Compression
% ------------------------------------------------------------
\begin{frame}{The Solution: Compress the Correlations!}
    
    \begin{center}
        \textbf{Key Insight:} Not all $2^L$ configurations are equally important.
    \end{center}
    
    \vspace{0.5cm}
    
    % === FIGURE PLACEHOLDER ===
    % Concept: Spectrum of singular values / eigenvalues
    %
    % ASCII draft:
    %   Eigenvalue spectrum: 
    %   λ₁ ████████████████  (dominant)
    %   λ₂ ████████████
    %   λ₃ ████████
    %   λ₄ █████
    %   λ₅ ███
    %   ...
    %   λₙ █                 (negligible)
    %
    %   Arrow: "Keep only χ largest!"
    \begin{center}
        \fbox{\parbox{0.6\textwidth}{
            \centering
            \vspace{2.5cm}
            \textit{[Figure: Eigenvalue spectrum decay --- keep only $\chi$ largest]}
            \vspace{2.5cm}
        }}
    \end{center}
    
    \vspace{0.3cm}
    
    \begin{block}{Strategy}
        \textbf{Renormalization Group:} Keep only the $\chi$ most \goodpoint{relevant} degrees of freedom!
    \end{block}
\end{frame}

% ============================================================
% PART 2: iDMRG as a Warm-up
% ============================================================
\section{Warm-up: iDMRG in 1D}

% ------------------------------------------------------------
% Slide 2.1: MPS and MPO
% ------------------------------------------------------------
\begin{frame}{MPS and MPO: The Language of 1D}
    
    \begin{columns}
        \begin{column}{0.5\textwidth}
            \textbf{Matrix Product State (MPS):}
            \[
            |\psi\rangle = \sum_{\{s\}} A^{s_1} A^{s_2} \cdots A^{s_N} |s_1 s_2 \cdots s_N\rangle
            \]
            
            % === FIGURE PLACEHOLDER ===
            % Concept: MPS tensor train
            %
            % ASCII draft:
            %   ○──[A]──○──[A]──○──[A]──○──[A]──○
            %      │       │       │       │
            %      s₁      s₂      s₃      s₄
            %
            % Horizontal lines: virtual bonds (dim χ)
            % Vertical lines: physical indices (dim d)
            \begin{center}
                \fbox{\parbox{0.9\textwidth}{
                    \centering
                    \vspace{1.5cm}
                    \textit{[Figure: MPS tensor diagram]}
                    \vspace{1.5cm}
                }}
            \end{center}
        \end{column}
        
        \begin{column}{0.5\textwidth}
            \textbf{Matrix Product Operator (MPO):}
            \[
            \hat{H} = \sum_{\{s,s'\}} W^{s_1 s'_1} W^{s_2 s'_2} \cdots
            \]
            
            % === FIGURE PLACEHOLDER ===
            % Concept: MPO applied to MPS
            %
            % ASCII draft:
            %   ○──[A]──○──[A]──○──[A]──○  (MPS)
            %      │       │       │
            %   ○──[W]──○──[W]──○──[W]──○  (MPO)
            %      │       │       │
            %
            % Show MPO as "sandwich" layer
            \begin{center}
                \fbox{\parbox{0.9\textwidth}{
                    \centering
                    \vspace{1.5cm}
                    \textit{[Figure: MPO applied to MPS]}
                    \vspace{1.5cm}
                }}
            \end{center}
        \end{column}
    \end{columns}
    
    \vspace{0.3cm}
    \begin{block}{Key Feature}
        Bond dimension $\chi$ controls both \textbf{accuracy} and \textbf{computational cost}.
    \end{block}
\end{frame}

% ------------------------------------------------------------
% Slide 2.2: iDMRG Setup
% ------------------------------------------------------------
\begin{frame}{iDMRG: Infinite System from Finite Environment}
    
    \textbf{Goal:} Represent an \highlight{infinite} chain using \highlight{finite} environment tensors.
    
    \vspace{0.5cm}
    
    % === FIGURE PLACEHOLDER ===
    % Concept: Left environment - new sites - right environment
    %
    % ASCII draft:
    %   ···[L]──○──○──[R]···
    %       │   │  │   │
    %      Left  New  Right
    %      Env  Sites  Env
    %
    %   [L] = compressed left half-chain (χ states)
    %   [R] = compressed right half-chain (χ states)
    %   ○ = new physical sites being added
    \begin{center}
        \fbox{\parbox{0.75\textwidth}{
            \centering
            \vspace{2.5cm}
            \textit{[Figure: iDMRG setup --- Left Env + New Sites + Right Env]}
            \vspace{2.5cm}
        }}
    \end{center}
    
    \vspace{0.3cm}
    
    \begin{columns}
        \begin{column}{0.5\textwidth}
            \textbf{Left Environment $L$:}
            \begin{itemize}
                \item Encodes infinite left half-chain
                \item Compressed to $\chi$ states
            \end{itemize}
        \end{column}
        \begin{column}{0.5\textwidth}
            \textbf{Right Environment $R$:}
            \begin{itemize}
                \item Encodes infinite right half-chain
                \item Compressed to $\chi$ states
            \end{itemize}
        \end{column}
    \end{columns}
\end{frame}

% ------------------------------------------------------------
% Slide 2.3: iDMRG Iteration - Grow
% ------------------------------------------------------------
\begin{frame}{iDMRG Iteration: Step 1 --- Grow (Add Sites)}
    
    % === FIGURE PLACEHOLDER ===
    % Concept: Adding new site pair to the system
    %
    % ASCII draft:
    %   Before:  [L]──○──○──[R]   (2-site unit cell)
    %   
    %   After:   [L]──○──○──○──○──[R]  (grown by absorbing MPO)
    %                    ↑
    %               New sites added
    %
    % This is analogous to "decimation" in real-space RG
    \begin{center}
        \fbox{\parbox{0.8\textwidth}{
            \centering
            \vspace{3cm}
            \textit{[Figure: Growing step --- absorb new site + MPO into environment]}
            \vspace{3cm}
        }}
    \end{center}
    
    \vspace{0.3cm}
    
    \begin{block}{Analogy to Real-Space RG}
        Adding sites $\approx$ \textbf{Decimation} in Kadanoff's block-spin RG
    \end{block}
\end{frame}

% ------------------------------------------------------------
% Slide 2.4: iDMRG Iteration - Truncate
% ------------------------------------------------------------
\begin{frame}{iDMRG Iteration: Step 2 --- Truncate (SVD)}
    
    \textbf{Problem:} After growing, bond dimension becomes $\chi \cdot d$ --- too large!
    
    \vspace{0.3cm}
    
    % === FIGURE PLACEHOLDER ===
    % Concept: SVD truncation of the center bond
    %
    % ASCII draft:
    %   [====== ψ ======]     (combined wavefunction)
    %           ↓ SVD
    %   [U]──[Σ]──[V†]        (singular value decomposition)
    %           ↓ truncate to χ
    %   [U']──[Σ']──[V'†]     (keep χ largest singular values)
    %
    % Show singular value spectrum with cutoff
    \begin{center}
        \fbox{\parbox{0.75\textwidth}{
            \centering
            \vspace{2.5cm}
            \textit{[Figure: SVD truncation --- keep $\chi$ largest singular values]}
            \vspace{2.5cm}
        }}
    \end{center}
    
    \vspace{0.3cm}
    
    \begin{columns}
        \begin{column}{0.6\textwidth}
            \textbf{SVD on center bond:}
            \[
            \Psi = U \Sigma V^\dagger \xrightarrow{\text{truncate}} U' \Sigma' V'^\dagger
            \]
        \end{column}
        \begin{column}{0.4\textwidth}
            \begin{block}{Analogy to RG}
                Truncation $\approx$ \textbf{Rescaling}
            \end{block}
        \end{column}
    \end{columns}
\end{frame}

% ------------------------------------------------------------
% Slide 2.5: iDMRG Convergence
% ------------------------------------------------------------
\begin{frame}{iDMRG: Convergence to Fixed Point}
    
    \begin{center}
        \textbf{Iterate:} Grow $\to$ Truncate $\to$ Grow $\to$ Truncate $\to \cdots$
    \end{center}
    
    \vspace{0.3cm}
    
    % === FIGURE PLACEHOLDER ===
    % Concept: Convergence diagram
    %
    % ASCII draft:
    %   Iteration:  1    2    3    4    5    ...   ∞
    %                ↓    ↓    ↓    ↓    ↓          ↓
    %   Energy:    ───────────────────────→  E_fixed
    %   
    %   Environment tensors converge to fixed point!
    \begin{center}
        \fbox{\parbox{0.7\textwidth}{
            \centering
            \vspace{2cm}
            \textit{[Figure: Convergence of energy/environment to fixed point]}
            \vspace{2cm}
        }}
    \end{center}
    
    \vspace{0.3cm}
    
    \begin{block}{RG Fixed Point}
        Converged $L^*, R^*$ = \goodpoint{Fixed point} of the RG transformation.
        
        These encode the \textbf{compressed infinite environment}.
    \end{block}
    
    \vspace{0.3cm}
    
    \textbf{Computing observables:}
    \[
    \langle O \rangle = \frac{\langle L^* | O | R^* \rangle}{\langle L^* | R^* \rangle}
    \]
\end{frame}

% ============================================================
% PART 3: CTMRG for 2D
% ============================================================
\section{CTMRG: Extending to 2D}

% ------------------------------------------------------------
% Slide 3.1: 2D Environment Structure
% ------------------------------------------------------------
\begin{frame}{2D Analog: Four Corners + Four Edges}
    
    \textbf{Key Idea:} Divide infinite 2D lattice into \highlight{4 corners} + \highlight{4 edges}.
    
    \vspace{0.3cm}
    
    % === FIGURE PLACEHOLDER ===
    % Concept: The full 2D environment structure
    %
    % ASCII draft:
    %        C₄ ─── T₄ ─── C₁
    %         │             │
    %        T₃      ●     T₁
    %         │             │
    %        C₃ ─── T₂ ─── C₂
    %
    % C = corner matrices (χ × χ)
    % T = edge tensors (χ × d × χ)
    % ● = local tensor (center)
    %
    % Color code: corners in one color, edges in another
    \begin{center}
        \fbox{\parbox{0.55\textwidth}{
            \centering
            \vspace{3.5cm}
            \textit{[Figure: 2D environment structure --- 4 corners + 4 edges]}
            \vspace{3.5cm}
        }}
    \end{center}
    
    \vspace{0.3cm}
    
    \begin{columns}
        \begin{column}{0.5\textwidth}
            \textbf{Corners $C_1, C_2, C_3, C_4$:}
            \begin{itemize}
                \item Encode quadrant contributions
                \item Size: $\chi \times \chi$
            \end{itemize}
        \end{column}
        \begin{column}{0.5\textwidth}
            \textbf{Edges $T_1, T_2, T_3, T_4$:}
            \begin{itemize}
                \item Half-infinite row/column
                \item Size: $\chi \times d \times \chi$
            \end{itemize}
        \end{column}
    \end{columns}
\end{frame}

% ------------------------------------------------------------
% Slide 3.2: Local Tensor
% ------------------------------------------------------------
\begin{frame}{The Local Tensor: Building Block}
    
    \begin{columns}
        \begin{column}{0.5\textwidth}
            \textbf{Local Boltzmann weight tensor $a$:}
            
            \vspace{0.5cm}
            
            % === FIGURE PLACEHOLDER ===
            % Concept: 4-leg tensor for one site
            %
            % ASCII draft:
            %              up
            %               │
            %    left ──── [a] ──── right
            %               │
            %             down
            %
            % Each leg connects to a neighbor
            \begin{center}
                \fbox{\parbox{0.8\textwidth}{
                    \centering
                    \vspace{2cm}
                    \textit{[Figure: Local tensor $a$ with 4 legs]}
                    \vspace{2cm}
                }}
            \end{center}
        \end{column}
        
        \begin{column}{0.5\textwidth}
            \textbf{For 2D Ising:}
            \[
            a_{u,d,l,r} = \sum_{\sigma} W_{\sigma,u} W_{\sigma,d} W_{\sigma,l} W_{\sigma,r}
            \]
            
            where
            \[
            W_{\sigma, \sigma'} = e^{\frac{\beta J}{2} \sigma \sigma'}
            \]
            
            \vspace{0.3cm}
            
            \begin{block}{Physical Meaning}
                $a$ encodes the Boltzmann weight at one site, with bond weights split symmetrically.
            \end{block}
        \end{column}
    \end{columns}
\end{frame}

% ------------------------------------------------------------
% Slide 3.3: CTMRG Grow Step
% ------------------------------------------------------------
\begin{frame}{CTMRG Iteration: Step 1 --- Grow (Absorb Row/Column)}
    
    \textbf{Absorption:} Add one row \textbf{and} one column to expand the environment.
    
    \vspace{0.3cm}
    
    % === FIGURE PLACEHOLDER ===
    % Concept: Corner absorption step
    %
    % ASCII draft (showing one direction):
    %   Before:           After absorption:
    %   
    %   C₁ ─── T₁         C₁'────────
    %    │      │    →     │         │
    %   T₄ ─── a          T₄'       T₁'
    %                      │         │
    %                      └─── a ───┘
    %
    % Show how corner "eats" the local tensor
    % Bond dimension grows: χ → χ·d
    \begin{center}
        \fbox{\parbox{0.8\textwidth}{
            \centering
            \vspace{3cm}
            \textit{[Figure: Absorption step --- corner grows by absorbing $a$ and edges]}
            \vspace{3cm}
        }}
    \end{center}
    
    \vspace{0.3cm}
    
    \begin{alertblock}{Bond Dimension Growth}
        After absorption: $C'$ has size $(\chi \cdot d) \times (\chi \cdot d)$ --- \highlight{too large!}
    \end{alertblock}
\end{frame}

% ------------------------------------------------------------
% Slide 3.4: CTMRG Truncate Step
% ------------------------------------------------------------
\begin{frame}{CTMRG Iteration: Step 2 --- Truncate (The Key Difference!)}
    
    \textbf{Truncation in 2D:} Use the \highlight{full environment} to determine projectors.
    
    \vspace{0.3cm}
    
    % === FIGURE PLACEHOLDER ===
    % Concept: How to compute projector in 2D
    %
    % ASCII draft:
    %   Build "half-system" density matrix:
    %   
    %        C₄' ─── T₄' ─── C₁'
    %         │               │
    %        T₃'             T₁'
    %         │               │
    %        ─────── ρ ───────
    %                ↓
    %         SVD/Eigendecomposition
    %                ↓
    %         Projector P (keep χ largest)
    %
    % This is different from 1D: need to consider 2D geometry!
    \begin{center}
        \fbox{\parbox{0.75\textwidth}{
            \centering
            \vspace{2.5cm}
            \textit{[Figure: Building density matrix from environment for truncation]}
            \vspace{2.5cm}
        }}
    \end{center}
    
    \vspace{0.3cm}
    
    \begin{columns}
        \begin{column}{0.55\textwidth}
            \textbf{Compute projector $P$:}
            \begin{enumerate}
                \item Contract half-environment $\to \rho$
                \item SVD: $\rho = U \Sigma V^\dagger$
                \item Keep $\chi$ largest: $P = U_{:\chi}$
            \end{enumerate}
        \end{column}
        \begin{column}{0.45\textwidth}
            \textbf{Apply truncation:}
            \[
            C_{\text{new}} = P^\dagger C' P
            \]
            \[
            T_{\text{new}} = P^\dagger T' P
            \]
        \end{column}
    \end{columns}
\end{frame}

% ------------------------------------------------------------
% Slide 3.5: CTMRG Full Algorithm
% ------------------------------------------------------------
\begin{frame}{CTMRG: Complete Algorithm}
    
    % === FIGURE PLACEHOLDER ===
    % Concept: Flowchart of CTMRG algorithm
    %
    % ASCII draft:
    %   ┌─────────────────────────────────────┐
    %   │ Initialize: C₁, C₂, C₃, C₄, T₁-T₄  │
    %   └───────────────┬─────────────────────┘
    %                   ↓
    %   ┌─────────────────────────────────────┐
    %   │ 1. Absorb: grow all corners & edges │◄───┐
    %   └───────────────┬─────────────────────┘    │
    %                   ↓                          │
    %   ┌─────────────────────────────────────┐    │
    %   │ 2. Compute projectors P from ρ      │    │
    %   └───────────────┬─────────────────────┘    │
    %                   ↓                          │
    %   ┌─────────────────────────────────────┐    │
    %   │ 3. Truncate: C_new = P† C' P        │    │
    %   └───────────────┬─────────────────────┘    │
    %                   ↓                          │
    %   ┌─────────────────────────────────────┐    │
    %   │ 4. Converged?  ──No──────────────────────┘
    %   └───────────────┬─────────────────────┘
    %                   │ Yes
    %                   ↓
    %   ┌─────────────────────────────────────┐
    %   │ Output: Fixed-point environment     │
    %   └─────────────────────────────────────┘
    \begin{center}
        \fbox{\parbox{0.7\textwidth}{
            \centering
            \vspace{4cm}
            \textit{[Figure: CTMRG algorithm flowchart]}
            \vspace{4cm}
        }}
    \end{center}
\end{frame}

% ------------------------------------------------------------
% Slide 3.6: Computing Observables
% ------------------------------------------------------------
\begin{frame}{Computing Observables with CTMRG}
    
    \textbf{Partition function:}
    \[
    Z \propto \text{Tr}(C_1 T_1 C_2 T_2 C_3 T_3 C_4 T_4)
    \]
    
    \vspace{0.3cm}
    
    % === FIGURE PLACEHOLDER ===
    % Concept: How to compute local magnetization
    %
    % ASCII draft:
    %        C₄ ─── T₄ ─── C₁
    %         │             │
    %        T₃      σ     T₁    ← insert σ at center
    %         │             │
    %        C₃ ─── T₂ ─── C₂
    %
    % Contract everything to get <σ>
    \begin{center}
        \fbox{\parbox{0.6\textwidth}{
            \centering
            \vspace{2.5cm}
            \textit{[Figure: Computing $\langle \sigma \rangle$ by inserting operator]}
            \vspace{2.5cm}
        }}
    \end{center}
    
    \vspace{0.3cm}
    
    \textbf{Local magnetization:}
    \[
    \langle \sigma_{\text{center}} \rangle = \frac{\text{Tr}(C_1 T_1 C_2 T_2 C_3 T_3 C_4 T_4 \cdot \sigma)}{Z}
    \]
\end{frame}

% ============================================================
% PART 4: Results
% ============================================================
\section{Results}

% ------------------------------------------------------------
% Slide 4.1: Results (Placeholder)
% ------------------------------------------------------------
\begin{frame}{Numerical Results: 2D Ising Model}
    
    % === RESULTS PLACEHOLDER ===
    % To be filled with actual numerical results
    % Suggestions:
    % - Magnetization vs temperature curve
    % - Comparison with exact Onsager solution
    % - Convergence with bond dimension χ
    % - Critical exponents
    
    \begin{center}
        \fbox{\parbox{0.85\textwidth}{
            \centering
            \vspace{5cm}
            \textit{[Results to be added: magnetization, critical behavior, etc.]}
            \vspace{5cm}
        }}
    \end{center}
    
\end{frame}

% ============================================================
% PART 5: Summary
% ============================================================
\section{Summary}

% ------------------------------------------------------------
% Slide 5.1: Core Steps
% ------------------------------------------------------------
\begin{frame}{Summary: The Core of CTMRG}
    
    \begin{block}{CTMRG in Three Steps}
        \begin{enumerate}
            \item \textbf{Decompose:} Infinite 2D lattice $\to$ 4 corners + 4 edges
            \item \textbf{Grow:} Absorb local tensors (add row/column)
            \item \textbf{Truncate:} SVD-based RG to keep $\chi$ most relevant states
        \end{enumerate}
    \end{block}
    
    \vspace{0.5cm}
    
    % === FIGURE PLACEHOLDER ===
    % Concept: Visual summary of the three steps
    %
    % ASCII draft:
    %   [Decompose]      [Grow]          [Truncate]
    %   4 corners    →   absorb a    →   SVD + keep χ
    %   + 4 edges        (larger)        (back to χ)
    %
    % Show as three-panel diagram
    \begin{center}
        \fbox{\parbox{0.85\textwidth}{
            \centering
            \vspace{2.5cm}
            \textit{[Figure: Visual summary --- Decompose $\to$ Grow $\to$ Truncate]}
            \vspace{2.5cm}
        }}
    \end{center}
\end{frame}

% ------------------------------------------------------------
% Slide 5.2: DMRG vs CTMRG
% ------------------------------------------------------------
\begin{frame}{Comparison: iDMRG vs CTMRG}
    
    \begin{table}
        \centering
        \renewcommand{\arraystretch}{1.3}
        \begin{tabular}{l|c|c}
            \toprule
            \textbf{Aspect} & \textbf{iDMRG (1D)} & \textbf{CTMRG (2D)} \\
            \midrule
            Dimension & 1D chain & 2D square lattice \\
            \midrule
            Environment & Left + Right & 4 Corners + 4 Edges \\
            \midrule
            Grow step & Add site pair & Add row + column \\
            \midrule
            Truncation & SVD on center bond & SVD on corner boundaries \\
            \midrule
            Bond dimension & $\chi$ (MPS) & $\chi$ (environment) \\
            \midrule
            Fixed point & $L^*, R^*$ & $C_i^*, T_i^*$ \\
            \midrule
            Computational cost & $O(\chi^3)$ & $O(\chi^6)$ or $O(\chi^5)$ \\
            \bottomrule
        \end{tabular}
    \end{table}
    
    \vspace{0.3cm}
    
    \begin{block}{Take-Home Message}
        \goodpoint{Same RG philosophy}, different geometric structure!
    \end{block}
\end{frame}

% ============================================================
% Thank You Slide
% ============================================================
\begin{frame}
    \begin{center}
        \Huge \textbf{Thank You!}
        
        \vspace{1cm}
        
        \Large Questions?
    \end{center}
\end{frame}

% ============================================================
% APPENDIX
% ============================================================
\appendix
\section*{Appendix}

% ------------------------------------------------------------
% Appendix A.1: No Sign Problem
% ------------------------------------------------------------
\begin{frame}{Appendix: No Monte Carlo Sign Problem}
    
    \textbf{Classical models:} All Boltzmann weights are \highlight{positive}!
    
    \[
    W = e^{-\beta H} > 0 \quad \text{always}
    \]
    
    \vspace{0.3cm}
    
    % === FIGURE PLACEHOLDER ===
    % Concept: Comparison of classical vs quantum MC
    %
    % ASCII draft:
    %   Classical:  W > 0  →  No sign problem  →  MC works!
    %   Quantum:    W can be < 0 (fermions, frustration)
    %                      →  Sign problem  →  MC fails
    \begin{center}
        \fbox{\parbox{0.7\textwidth}{
            \centering
            \vspace{2cm}
            \textit{[Figure: Classical (positive) vs Quantum (sign problem)]}
            \vspace{2cm}
        }}
    \end{center}
    
    \vspace{0.3cm}
    
    \begin{block}{Advantage of Tensor Network Methods}
        \begin{itemize}
            \item No statistical sampling $\Rightarrow$ no sign problem issue
            \item Deterministic algorithm with controlled error (via $\chi$)
            \item Works equally well for frustrated/quantum systems
        \end{itemize}
    \end{block}
\end{frame}

% ------------------------------------------------------------
% Appendix A.2: Why Not 3D?
% ------------------------------------------------------------
\begin{frame}{Appendix: Why is 3D Difficult?}
    
    \textbf{In 2D:} Environment tensors are \highlight{1D objects} (edges).
    
    \textbf{In 3D:} Environment would be \highlight{2D surfaces} --- back to exponential!
    
    \vspace{0.3cm}
    
    % === FIGURE PLACEHOLDER ===
    % Concept: 2D boundary in 3D vs 1D boundary in 2D
    %
    % ASCII draft:
    %   2D system:  boundary is 1D line (manageable with χ)
    %   3D system:  boundary is 2D surface (exponential again!)
    %
    % Show schematic of 3D cube with 2D boundary
    \begin{center}
        \fbox{\parbox{0.75\textwidth}{
            \centering
            \vspace{2.5cm}
            \textit{[Figure: 2D boundary problem in 3D systems]}
            \vspace{2.5cm}
        }}
    \end{center}
    
    \vspace{0.3cm}
    
    \begin{alertblock}{The 3D Challenge}
        No simple analog of CTMRG for 3D --- active research area!
        
        Possible approaches: Higher-order tensor RG, Monte Carlo + TN hybrids
    \end{alertblock}
\end{frame}

% ------------------------------------------------------------
% Appendix A.3: Extensions
% ------------------------------------------------------------
\begin{frame}{Appendix: Extensions and Generalizations}
    
    \textbf{Beyond square lattice Ising:}
    
    \vspace{0.3cm}
    
    \begin{columns}
        \begin{column}{0.5\textwidth}
            \textbf{Different lattices:}
            \begin{itemize}
                \item Honeycomb
                \item Triangular
                \item Kagome
            \end{itemize}
            
            \vspace{0.3cm}
            
            \textbf{Different models:}
            \begin{itemize}
                \item Potts model
                \item Clock model
                \item Vertex models
            \end{itemize}
        \end{column}
        
        \begin{column}{0.5\textwidth}
            \textbf{Quantum systems (via iPEPS):}
            \begin{itemize}
                \item 2D Heisenberg model
                \item Frustrated magnets
                \item Topological phases
            \end{itemize}
            
            \vspace{0.3cm}
            
            \textbf{Improvements:}
            \begin{itemize}
                \item Directional CTMRG
                \item Full-update vs simple-update
                \item Gradient optimization
            \end{itemize}
        \end{column}
    \end{columns}
\end{frame}

% ------------------------------------------------------------
% Appendix A.4: Computational Complexity
% ------------------------------------------------------------
\begin{frame}{Appendix: Computational Complexity}
    
    \textbf{CTMRG scaling:}
    
    \begin{table}
        \centering
        \begin{tabular}{l|c|c}
            \toprule
            \textbf{Operation} & \textbf{Cost} & \textbf{Bottleneck?} \\
            \midrule
            Corner absorption & $O(\chi^4 d^2)$ & \\
            Edge absorption & $O(\chi^3 d^2)$ & \\
            Build density matrix & $O(\chi^4)$ & \\
            SVD for projector & $O(\chi^3 d^3)$ & \checkmark \\
            Apply truncation & $O(\chi^3 d)$ & \\
            \midrule
            \textbf{Total per iteration} & $O(\chi^3 d^3)$ to $O(\chi^6)$ & \\
            \bottomrule
        \end{tabular}
    \end{table}
    
    \vspace{0.3cm}
    
    \begin{block}{Comparison}
        \begin{itemize}
            \item iDMRG (1D): $O(\chi^3)$ per sweep
            \item CTMRG (2D): $O(\chi^5)$ -- $O(\chi^6)$ per iteration
            \item More expensive, but still \goodpoint{polynomial} in $\chi$!
        \end{itemize}
    \end{block}
\end{frame}

% ------------------------------------------------------------
% Appendix A.5: Key References
% ------------------------------------------------------------
\begin{frame}{Appendix: Key References}
    
    \textbf{Original works:}
    \begin{itemize}
        \item R. J. Baxter, \textit{J. Math. Phys.} \textbf{9}, 650 (1968) --- Corner transfer matrices
        \item R. J. Baxter, \textit{J. Stat. Phys.} \textbf{19}, 461 (1978) --- CTM method
    \end{itemize}
    
    \vspace{0.3cm}
    
    \textbf{Modern CTMRG:}
    \begin{itemize}
        \item T. Nishino \& K. Okunishi, \textit{J. Phys. Soc. Jpn.} \textbf{65}, 891 (1996)
        \item T. Nishino \& K. Okunishi, \textit{J. Phys. Soc. Jpn.} \textbf{66}, 3040 (1997)
    \end{itemize}
    
    \vspace{0.3cm}
    
    \textbf{CTMRG for iPEPS:}
    \begin{itemize}
        \item R. Orús \& G. Vidal, \textit{Phys. Rev. B} \textbf{80}, 094403 (2009)
        \item P. Corboz et al., \textit{Phys. Rev. B} \textbf{84}, 041108(R) (2011)
    \end{itemize}
    
    \vspace{0.3cm}
    
    \textbf{Reviews:}
    \begin{itemize}
        \item R. Orús, \textit{Ann. Phys.} \textbf{349}, 117 (2014) --- Tensor networks review
    \end{itemize}
\end{frame}

\end{document}
